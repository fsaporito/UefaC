\documentclass[../uefaC.tex]{subfiles}

\subfilecover{Saporito Francesco}{Giustizia Sportiva}

\begin{document}

\subfileintro{Giustizia Sportiva (Parte di Carte Federali)}

Durata Corso: 2h (Teoria in Aula) \hfill \\
Esame: Quiz a Crocette (Assieme a Carte Federali)
\hfill \\
\hfill \\
\hfill \\
In questo corso affrontiamo i punti salienti del codice di Giustizia Sportiva della FIGC nella versione emanata nel 2019. \hfill \\
Questo codice va ad integrarsi innanzitutto con le N.O.I.F., ma anche con tutte le altre fonti normative (in particolare codice civile e codice penale italiani).
\hfill \\

\section{Giudice Sportivo}
In particolare il \emph{Codice di Giustizia Sportiva} tutela in primis l'\textbf{Arbitro} e le sue funzioni. Infatti al \emph{Giudice Sportivo} arriva soltanto il referto di gara scritto dall'arbitro e non è ammesso su di esso il diritto di replica o di difesa. Il referto arbitrale è dunque una \emph{Fonte Privilegiata di Prova} ed è l'unico elemento sul quale vengono in prima istanza dal giudice sportivo territoriale (provinciale). Da notare inoltre che il giudice sportivo segue soltanto le questioni / problematiche / eventi delle partite di gara e demanda il resto alla \emph{Procura Federale} che ha altre competenze e poteri. \hfill \\
Le decisioni del Giudice Sportivo vengono messe in atto dalla mezzanotte del giorno successivo all'emissione del comunicato ufficiale (di solito il giovedi).

\section{Il Ricorso}
Le società, una volta visionata sul comunicato ufficiale la decisione presa dal giudice sportivo territoriale, possono fare ricorso alla corte sportiva di appello regionale (in Lombardia a Milano), la quale ha il potere di riformare la decisione presa, sia aumentando che riducendo la pena (a differenza della giustizia ordinaria dove i gradi superiori possono solo confermare o ridurre la pena). \hfill \\
Inoltre le sanzioni inferiori ai 30 giorni di squalifica o di massimo 2 giornate di squalifica sono ritenute inappellabili, ovvero non è possibile fare ricorso.

\section{Procura Federale e Tribunale Federale}
Enti diversi dalla corte sportiva di appello (regionale o nazionale) sono \emph{La procura Federale} che si occupa di tutto ciò che esterno alla normativa associata alla singola partita giocata (ad esempio gli illeciti sportivi) e il \emph{Tribunale Federale}, il quale si occupa di gestire i rapporti e le eventuali problematiche tra i tesserati (calciatori, dirigenti, tecnici) e le società.

\section{Articoli Principali}
Vediamo in questa sezione alcuni articoli della giustizia sportiva. In particolare è utile specificare che le \underline{Squalifiche} possono essere sia a giornate che a tempo. La differenza è che le squalifiche a giornate valgono solo per il campionato nel quale il tesserato ha subito la sanzione mentre la squalifica a tempo vale come una totale inibizione alla presenza nel recinto di gioco per qualunque campionato durante gare ufficiali.

\subsection{Art. 9: Sanzioni a Dirigenti, Soci e Tesserati}
Definisce il tipo di sanzioni che possono essere applicati a tutti i vari tesserati di una società calcistica:
\begin{itemize}
    \item Ammonizione
    \item Ammonizione con Diffida
    \item Ammenda
    \item Ammenda con Diffida
    \item Squalifica per una o più giornate di gara (nello stesso campionato)
    \item Squalifica a tempo determinato (in ogni gara ufficiale della propria società)
    \item Divieto temporaneo di accesso agli impianti sportivi
    \item Inibizione temporanea a svolgere qualunque attività in ambito FIGC
\end{itemize}
\subsection{Art. 11: Sanzioni sulla Disputa della Gara}

\subsection{Art. 13: Circostanze Attenuanti}
Stabilisce tutte le circostanze attenuanti una specifica sanzione, tra cui l'l'aver reagito in reazione ad un comportamento ingiusto altrui, all'essersi adoperato per attenuare le conseguenze dannose o aver ammesso la propria responsabilità.

\subsection{Art. 14: Circostanze Aggravanti}
Stabilisce tutte le circostanze aggravanti una specifica sanzione.

\subsection{Art. 15: Concorso di Circostanze}
Stabilisce come la concorrenza di multiple circostanti (aggravanti, attenuanti o entrambe) vanno ad influenzare sulle sanzioni.

\subsection{Art.28: Comportamento Discriminatorio}
Definisce il comportamento discriminatorio e le derivanti sanzioni. In particolare un \emph{Comportamento Discriminatorio} è definito come ogni condotta che, direttamente o indirettamente, comporta offesa, denigrazione o insulto per motivi di razza, colore, religione, lingua, sesso, nazionalità, origine anche etnica, condizione personale o sociale ovvero configura propaganda ideologica vietta dalla legge o comunque inneggiante a comportamenti discriminatori.

\subsection{Art. 35: Condotta Violenta nei Confronti degli Ufficiali di Gara}
Questo articolo definisce le pene nei confronti di atleti, dirigenti e tecnici che vanno ad effettuare una condotta violenta nei contronti di uno qualunque degli ufficiali di gara. La condotta violenta si intende come ogni atto intenzionale diretto a produrre una lesione personale. \hfill \\
In particolare si valuta non solo il danno fisico effettivamente sostenuto ma soprattuto l'intenzione nel fare male. La qualifica minima per i dirigenti è di un anno mentre è di 5 giornate per tenici e calciatori (la squalifica viene potenzialmente raddoppiata in caso di presenza di referti medici).
\hfill \\
Inoltre è importante notare che nella giustizia sportiva calcistica è presente il concetto di \underline{Responsabilità Oggettiva} delle società, che sono indicate come responsabili delle azioni dei propri tesserati (tecnici, atleti, dirigenti, soci) e dei propri sostenitori. \hfill \\

\subsection{Art. 36: Altre Condotte nei Confronti degli Ufficiali di Gara}
Definisce le condotte ingiuriose o irriguardose (almeno due giornate di squalifica o a tempo determinato) e le condotte gravemente irriguardose che si concretizzano in un contatto fisico non rientrante nella condotta violenta (almeno 4 giornate o a tempo determinato).

\subsection{Art. 37: Utilizzo di Espressioni Blasfeme}
In caso di espressione blasfema, viene inflitta ai calciatori e ai tecnici la sanzione minima della squalifica di una giornata mentre agli altri soggetti ammessi nel recinto di gioco viene inflitta la sanzione dell'inibizione.

\subsection{Art. 38: Condotta Violenta nei dei Calciatori}
Stabilisce che i calciatori colpevoli di condotta violenta verso altri calciatori o altre persone nel contesto di una gara vengono sanciti con la sanzione minima di 3 giornate di squalifica o a tempo determinato, al netto di circostanze aggravanti o attenuanti.


\end{document}
