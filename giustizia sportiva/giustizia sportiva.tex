\documentclass[../uefaC.tex]{subfiles}

\subfilecover{Saporito Francesco}{Giustizia Sportiva}

\begin{document}

\subfileintro{Giustizia Sportiva (Parte di Carte Federali)}

Durata Corso: 2h (Teoria in Aula) \hfill \\
Esame: Quiz a Crocette (Assieme a Carte Federali)
\hfill \\
\hfill \\
\hfill \\
In questo corso affrontiamo i punti salienti del codice di Giustizia Sportiva della FIGC nella versione emanata nel 2019. \hfill \\
Questo codice va ad integrarsi innanzitutto con le N.O.I.F., ma anche con tutte le altre fonti normative (in particolare codice civile e codice penale italiani).
\hfill \\

\subsection{Giudice Sportivo}
In particolare il \emph{Codice di Giustizia Sportiva} tutela in primis l'\textbf{Arbitro} e le sue funzioni. Infatti al \emph{Giudice Sportivo} arriva soltanto il referto di gara scritto dall'arbitro e non è ammesso su di esso il diritto di replica o di difesa. Il referto arbitrale è dunque una \emph{Fonte Privilegiata di Prova} ed è l'unico elemento sul quale vengono in prima istanza dal giudice sportivo territoriale (provinciale). Da notare inoltre che il giudice sportivo segue soltanto le questioni / problematiche / eventi delle partite di gara e demanda il resto alla \emph{Procura Federale} che ha altre competenze e poteri. \hfill \\
Le decisioni del Giudice Sportivo vengono messe in atto dalla mezzanotte del giorno successivo all'emissione del comunicato ufficiale (di solito il giovedi).

\subsection{Il Ricorso}
Le società, una volta visionata sul comunicato ufficiale la decisione presa dal giudice sportivo territoriale, possono fare ricorso alla corte sportiva di appello regionale (in Lombardia a Milano), la quale ha il potere di riformare la decisione presa, sia aumentando che riducendo la pena (a differenza della giustizia ordinaria dove i gradi superiori possono solo confermare o ridurre la pena).



\end{document}
