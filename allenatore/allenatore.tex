\documentclass[../uefaC.tex]{subfiles}

\subfilecover{Saporito Francesco}{La Figura dell'Allenatore}

\begin{document}

\subfileintro{La Figura dell'Allenatore (Parte di Tecnica e Tattica Calcistica)}

L'allenatore è la figura cardine dell'apprendimento calcistico, è infatti colui che \underline{Educa} partendo dal giocatore stesso. L'allenatore deve essere appassionato per lo sport che insegna e deve essere particolarmente predisposto al rapporto con ragazzi e ragazze dell'età che ha scelto di allenare. \hfill \\
In particolare l'allenatore deve essere paziente sia con se stesso che con gli atleti, aspettando l'apprendimento e la crescità nei giusti tempi senza forzare le tappe. \hfill \\
E' inoltre fondamentale per l'allenatore essere capace di produrre situazioni positive (ambiente sano) e attive dove i calciatori siano i protagonisti dell'esperienza e del percorso di apprendimento. \hfill \\
L'allenatore deve saper ricoprire diversi ruoli:
\begin{enumerate}
    \item\textbf{Animatore}: Deve saper far divertire e Creare Gruppo.
    \item\textbf{Educatore}: Deve poter trasmettere dei valori.
    \item\textbf{Psicologo}: Deve essere in grado di interfacciarsi relazionalmente con gli atleti e capirne eventuali difficoltà emotivo-relazionali.
    \item\textbf{Insegnante}: Deve saper insegnare.
    \item\textbf{Tecnico}: Deve conoscere lo sport per cui allena.
    \item\textbf{Organizzatore}: Deve organizzare gli allenamenti, la squadra, le partite, ...
\end{enumerate}
Le competenze che deve avere un allenatore sono le seguenti:
\begin{enumerate}
    \item\textbf{Metodologiche}: Deve saper da un lato programmare sul medio-lungo periodo ma dall'altro anche essere in grado di capire quando e come modificare questa programmazione in base ai risultati parziali ottenuti. Deve inoltre saper spiegare le attività proposte usando stili di insegnamento sia partecipativi che direttivi in base al contesto. Deve poi inoltre essere in grado di continuamente variare il metodo di allenamento e le proposte, anche a fronte dello stesso obbiettivo, per mantenere alta l'attenzione degli atleti e per non scendere in automatismi al fine di sviluppare anche gli aspetti cognitivi. Infine deve verificare i risultati dell'apprendimento ottenuti rispetto agli obbiettivi prefissati.
    \item\textbf{Tecniche}: Deve conoscere sia i gesti tecnici da allenare che il modo e i tempi in cui utilizzarli e saperli dimostrare. Deve poi conoscere e gestire i fattori della prestazione degli atleti e svilupparli tutti in modo armonioso.
    \item\textbf{Organizzative}: Deve sapere orgaizzare la seduta di allenamento rispetto all'obbiettivo prefissato, gestendo attrezzi, spazi e tempi in modo funzionale sia all'obbiettivo che all'età e al livello dei giocatori. Deve inoltre programmare le attività gestendo le tempistiche corrette, valutando i tempi di riposo e riducendo i tempi morti tenendo alta l'intensità.
    \item\textbf{Relazionali}: Deve creare un ambiente accogliente, sano e sereno utilizzando un linguaggio corretto, semplice e sviluppando un modello di comunicazione tra pari. Deve essere autorevole ma quando serve anche autoritario, stabilendo un sistema di regole al fine di garantire la convivenza e la prosperità del gruppo (regole che lui stesso deve essere il primo a dover rispettare).
\end{enumerate}
\end{document}