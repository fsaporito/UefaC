\documentclass[../uefaC.tex]{subfiles}

\subfilecover{Saporito Francesco}{Tecnica e Tattica Calcistica}

\begin{document}

\subfileintro{Tecnica e Tattica Calcistica}

Durata Corso: 58h (34h Teoria in Aula, 24h Pratica sul Campo) \hfill \\
Esame: Orale + Pratico in Campo

\textcolor{red}{TODO}

\section{Tattica Collettiva}

La \textbf{Tattica Collettiva} è un movimento \underline{Predeterminato} di \underline{2 o più giocatori}, di un reparto o di una squadra mirato ad ottenere un \textbf{Obbiettivo} offensivo o difensivo. \hfill \\
In particolare nel gioco del calcio dividiamo il gioco in quattro fasi su cui poi andiamo ad articolare la tattica:
\begin{enumerate}
    \item{Fase di Possesso}: La mia squadra ha il possesso del Pallone.
    \item{Fase di Non Possesso}: La squadra avversaria ha il possesso del Pallone.
    \item{Transizione Positiva}: La mia squadra riconquista il possesso del Pallone. (Da Non Possesso a Possesso)
    \item{Transizione Negativa}: La mia squadra perde il possesso del Pallone. (Da Possesso a Non Possesso)
\end{enumerate}

\section{Principi di Gioco}

Consideriamo in questa sezione i principi di gioco sottostanti la tattica collettiva. Un \underline{Principio} in particolare differisce da una regola in quanto la regola prescrive un comportamento da svolgere e quindi un dettame a cui bisogna ubbedire, mentre un principio indica un valore / idea interiorizzata di quello che si vuole fare senza dare dettami precisi su come agire. Dividiamo quindi i principi in due categorie, in base a se siamo in fase di possesso o di non possesso:
\begin{table}[]
    \begin{tabular}{ll}
        \textbf{Principi in Fase di Possesso} & $\;$ \textbf{Principi in Fase di Non Possesso} \\
        & \\
        $\;$ $\; \cdot$ Scaglionamento offensivo              & $\;$ $\; \cdot$ Scaglionamento difensivo         \\
        $\;$ $\; \cdot$ Verticalizzazione                     & $\;$ $\; \cdot$ Azione ritardatrice              \\
        $\;$ A$\; \cdot$ mpiezza                              & $\;$ $\; \cdot$ Concentrazione                   \\
        $\;$ $\; \cdot$ Mobilità                              & $\;$ $\; \cdot$ Equilibrio                       \\
        $\;$ $\; \cdot$ Imprevedibilità                       & $\;$ $\; \cdot$ Controllo e cautela             
    \end{tabular}
\end{table}
\hfill \\ 
Da notare in particolare che i principi di Possesso e di Non Possesso sono posizionati nella tabella sulla stessa riga in \textbf{Contrapposizione} tra di loro. Ad esempio allo \underline{Scaglionamento Offensivo} si contrappone lo \underline{Scaglionamento Difensivo}.

\subsection{Scaglionamento Offensivo}

\subsection{Verticalizzazione}

\subsection{Ampiezza}

\subsection{Mobilità}

\subsection{Imprevedibilità}

\subsection{Scaglionamento Difensivo}

\subsection{Azione Ritardatrice}

\subsection{Concentrazione}

\subsection{Equilibrio}

\subsection{Controllo e Cautela}

\section{Sviluppi di Gioco in Fase di Possesso}

\subsection{Triangolazione}

\subsection{Taglio}

\subsection{Gioco su Passante}

\subsection{Sovrapposizione}

\subsection{Attacco Diretto (1 o 2 Tempi)}

\subsection{Tiro in Porta}

\subsection{Traversone}

\subsection{Cross}

\subsection{Mantenimento del Possesso}

\subsection{Cambio di Gioco}

\subsection{Esca}

\subsection{Velo}

\section{Sviluppi di Gioco in Fase di Non Possesso}

\subsection{Marcatura}

\subsection{Pressione}

\subsection{Pressing}

\subsection{Fuorigioco}

\subsection{Elastico Difensivo}

\subsection{Raddoppio del Duelllo}









\end{document}