\documentclass[../uefaC.tex]{subfiles}

\subfilecover{Saporito Francesco}{Carte Federali}

\begin{document}

\subfileintro{Carte Federali}

\hfill \\
Durata Corso: 2h (Teoria in Aula) \hfill \\
Esame: Quiz a Crocette (Assieme a Giustizia Sportiva)

\hfill \\
\section{Introduzione}
Le carte federali sono le fonti dell'Ordinamento Federale (Art. 2 VI co. S.F.). In particolare si possono identificare:
\begin{enumerate}
    \item Lo Statuto della Federazione Italiana Gioco Calcio (F.I.G.C.)
    \item Le Norme Organizzative Interne Federali (N.O.I.F.), il Codice di Giustizia Sportiva e le altre disposizioni emanate dal consiglio federale.
    \item Gli Statuti e i Regolamenti delle Leghe, delle Componenti Tecniche, dell'Associazione Italiana Arbitri (A.I.A.), del Settore Tecnico e del Settore Giovanile.
\end{enumerate}
\hfill \\
In particolare lo scopo dello studio delle carte federali è molteplice:
\begin{enumerate}
    \item Per esercitare al meglio la propria attività da Allenatore, il cui ruolo e funzione sta cambiando e richiede una sempre maggiore consapevolezza del contesto in cui opera.
    \item Per esercitare al meglio i propri diritti e conoscere i propri doveri.
    \item Per porre le basi per esercitare in futuro carriere differenti nel mondo del calcio.
\end{enumerate}

\section{Organizzazione FIGC}

La Federazione Italiana Gioco Calcio (F.I.G.C.) svolge le proprie funzioni, in piena autonomia tecnica e di gestione, in armonia con le deliberazioni e gli indirizzi della FIFA (Fédération Internationale de Football Association), della UEFA (Union of European Football Associations), del CIO (Comitato Olimpico Internazionale) e del CONI (Comitato Olimpico Nazionale Italiano). \hfill \\
L'organizzazione centrale della FIGC è composta dai seguenti ruoli / organi:
\begin{enumerate}
    \item L'Assemblea
    \item Il Presidente
    \item I Vice-Presidenti
    \item Il Comitato di Presidenza
    \item IL Consiglio Federale
    \item Il Direttore generale
    \item Il Collegio di dei Revisori dei Conti
\end{enumerate}
\hfill \\
La FIGC costituisce inoltre una propria organizzazione periferica secondo norme approvate dal Consiglio Federale. Fino a tale costituzione, i Presidenti dei Comitati Regionali e i delegati Provinciali della Lega Nazionale Dilettanti (LND) esercitano le funzioni rappresentative della FIGC ad essi delegate dal Consiglio Federale o dal Presidente Federale nei rapporti con le rispettive strutture periferiche del CONI, nonché in eventuali altri compiti di rappresentanza federale nel territorio di competenza, fatta salva la eventuale diversa delega. \hfill \\
Sono parte dell'Assemblea le seguenti componenti:
\begin{enumerate}
    \item Lega Nazionale Professionisti Serie A
    \item Lega Nazionale Professionisti Serie B
    \item Lega Calcio Professionistico
    \item Lega Nazionale Dilettanti (LND)
    \item Associazione Italiana Calciatori (AIC)
    \item Associazione Italiana Allenatori (AIAC)
    \item Associazione Italiana Arbitri (AIA)
\end{enumerate}
La FIGC è poi composta da due settori:
\begin{enumerate}
    \item{Settore Giovanile e Scolastico (SGS)} Promuove, discliplina e organizza l'attività dei giovani calciatori, con finalità tecniche, didattiche e sociali. Definisce in particolare l'attività calcistica giovanile in 3 diverse fasi:
    \begin{itemize}
        \item{Attività di Base} (Età dai 5 ai 12 anni)
        \item{Attività Agonistica} (Età dai 12 ai 16 anni)
        \item{Attività Scolastica}
    \end{itemize}
    \item{Settore Tecnico} E' l'organo di servizio della FIGC incaricato a norma dell'art. 14 dello statuto di svolgere attività di studio e qualificazione per la diffusione e il miglioramento della tecnica del gioco del calcio.
\end{enumerate}
\hfill \\
e da due divisioni:
\begin{enumerate}
    \item{Calcio Femminile} Si occupa della gestione ed organizzazione di tutti i campionait femminli. In particolare:
    \begin{itemize}
        \item{FIGC}: Si occupa della strategia di promozione e sviluppo del settore, dell'attività delle nazionali (Nazionale A, U23, U19, U17, U16) e dell'organizzazione dei campionati di Serie A e Serie B, del campionato Primavera e della Coppa Italia.
        \item{LND}: Si occupa dell'organizzazione del Campionato Nazionale di Serie C e dei Campionati Regionali di Eccellenza Femminile, Promozione Femminile e della Coppa Italia Regionale.
    \end{itemize}
    \item{Calcio a 5}: Si Occupa della promozione e dello sviluppo del settore, oltre all'organizzazione dei campionati nazionali (Serie A, Serie A2, Serie B, Serie A femminile, Serie A2 femminile), regionali e provinciali (Serie C1, Serie C2, Serie D), Campionato Nazionale U21 (Maschile), Campionato Nazionale U19 (sia maschile che femminile) e la Coppa Italia (sia maschile che femminile).
\end{enumerate}

\section{N.O.I.F}
Le \textbf{N.O.I.F.} (Norme Organizzative Interne Federali) sono tutte le 
I soggetti inclusi nelle N.O.I.F. sono:
\begin{enumerate}
    \item\textbf{La F.I.G.C} (Artt. 1 - 13)
    \item\textbf{Le Società} (Artt. 14 - 23)
    \item\textbf{Le Leghe} (Art. 24)
    \item\textbf{I Settori} (Art. 25)
    \item\textbf{L'A.I.A.} (Art. 26)
    \item\textbf{I Calciatori} (Artt. 27 - 35)
\end{enumerate}
mentre le funzioni sono:
\begin{enumerate}
    \item\textbf{Il Tesseramento} (Artt. 36 - 42)
    \item\textbf{Le Tutele} (Artt. 43 - 46)
    \item\textbf{Ordinamento dei Campionati e delle Gare} (Artt. 47 - 70)
    \item\textbf{Discliplina dei Calciatori in Campo} (Artt. 71 - 75)
    \item\textbf{Controlli sulla gestione economica finanziaria delle Società Professionistiche} (Artt. 76 - 90)
    \item\textbf{Rapporti tra le Società e i Calciatori} (Artt. 91 - 118)
\end{enumerate}
in particolare ci focalizziamo solo su alcuni aspetti delle N.O.I.F.

\subsection{Le Categorie dei Calciatori}
I calciatori vengono divisi in varie categorie:
\begin{itemize}
    \item\textbf{Professionisti} (Art. 28) Calciatori e Calciatrici che esercitano l'attività sportiva a titolo oneroso e con carattere di continuità, tesserati per una Società Professionistica Maschile (Serie A, B e C) o Femminile (Serie A, dalla stagione 2022-2023).
    \item\textbf{Non Professionisti} (Art. 29) Calciatori e Calciatrici tesserati che svolgono attività sportiva per società LND o che praticano il calcio a 5 o attività ricreativa (Art. 30)
    \item\textbf{Giovani} (Art. 31) Calciatori e Calciatrici che abbiano compiuto l'8° anno di età e che al 1° Gennaio dell'anno in cui ha inizio la stagione sportiva non abbiano compiuto il 16° anno di età. Il loro tesseramento è valido per una sola stagione sportiva.
    \item\textbf{Giovani Dilettanti} (Art. 32) I giovani dal compimento del 14° anno di età possono assumere il vincolo di tesseramento per la società dilettantistica presso cui sono già tesserati fino alla stagione sportiva entro la quale abbiano compiuto il 25° anno di età. Essi assumono al compimento del 18° anno la qualifica di "Non Professionista".
    \item\textbf{Giovani di Serie} (Art. 33) I giovani dal compimento del 14° anno di età acquisiscono la qualifica di "Giovani di Serie" quando sottoscrivono la richiesta di tesseramento per una società associata in una delle leghe professionistiche. Essi assumono un vincolo con sudetta società professionistica fino al termine della stagione in cui compiono il 19° anno, allo scopo di permettere alla società di addestrarli e prepararli ai campionati da essa disputati. La società ha diritto di stipulare con il giocatore il primo contratto da "Professionista", di durata massimo triennale.
\end{itemize}

\subsection{Il Tesseramento}
Il tesseramento si basa su tre principi fondamentali, ovvero \textbf{Obbligatorietà}, \textbf{Annualità} e \textbf{Unicità}. Esso riguarda i seguenti soggetti:
\begin{itemize}
    \item I Dirigenti Federali
    \item Gli Arbitri
    \item I Dirigenti e i Collaboratori nella Gestione Sportiva delle Società
    \item I Tecnici
    \item I Calciatori
\end{itemize}
Essendo il tesseramento unico, ne deriva che non si può essere tesserati in due società diverse, anche con ruoli differenti (es calciatore e dirigente). Ci sono però delle deroghe:
\begin{itemize}
    \item I Tecnici tesserati con società di Serie A e B con incarico differente da quello di \emph{Allenatore Responsabile della Prima Squadra} possono, previa risoluzione consensuale del contratto economico, essere autorizzati dal Settore Tecnico ad effettuare un secondo tesseramento nella stessa stagione sportiva come \textbf{Allenatore Responsabile della Prima Squadra} presso un'altra società di Serie A o B.
    \item Esclusione della Squadra dal Campionato.
    \item Trasferimenti all'Estero (previa risoluizione del contratto in essere
    \item I Tecnici non Professionisti possono essere tesserati come calciatori per la società in cui già sono tesserati come tecnici.
    \item I preparatori Atletici, i Medici Sociali e gli Operatori Sanitari che abbiano risolto il loro contratto e vogliano tesserarsi nella stessa stagione per un'altra società con la stessa mansione.
    \item Gli allenatori tesserati presso una società LND o di Puro Settore Giovanile hanno la facoltà, se esonerati entro il 30 Novembre della stagione sportiva, di tesserarsi presso un'altra società a condizione che la nuova società partecipi ad un girone o ad un campionato diverso dalla società che ha esonerato il tecnico. Questa deroga non è valida per gli allenatori esonerati dalla conduzione di squadre di attività di base. (Introdotta dal 2022)
\end{itemize}


\subsection{Ordinamento Campionati}
I Campionati Maschili di calcio a 11 sono divisi in:
\begin{enumerate}
    \item\textbf{Serie A} (20 Squadre, Nazionale, Professionistico)
    \item\textbf{Serie B} (20 Squadre, Nazionale, Professionistico)
    \item\textbf{Serie C} (3 Gironi da 20 Squadre, Nazionale, Professionistico)
    \item\textbf{Serie D} (9 Gironi, Nazionale, Semi-Professionistico)
    \item\textbf{Eccellenza} (Regionale, Dilettantistico)
    \item\textbf{Promozione} (Regionale, Dilettantistico)
    \item\textbf{1° Categoria} (Regionale, Dilettantistico)
    \item\textbf{2° Categoria} (Provinciale o Interprovinciale, Dilettantistico)
    \item\textbf{3° Categoria} (Provinciale, Dilettantistico)
\end{enumerate}
mentre Campionati Femminili di calcio a 11 sono divisi in:
\begin{enumerate}
    \item\textbf{Serie A} (Nazionale, Professionistico)
    \item\textbf{Serie B} (Nazionale, Dilettantistico)
    \item\textbf{Serie C} (Nazionale, Dilettantistico)
    \item\textbf{Eccellenza} (Regionale, Dilettantistico)
    \item\textbf{Promozione} (Regionale, Dilettantistico)
\end{enumerate}


\section{Settore Giovanile Scolastico (S.G.S)}
Il settore giovanile Scolastico si occupa dell'organizzazione, promozione e formazione di tutta l'attività giovanile. In particolare gestisce:
\begin{itemize}
    \item L'Attività di Base;
    \item L'Attività Giovanile Agonistica;
    \item L'Attività di Calcio Femminile Giovanile;
    \item L'Attività di Calcio a 5 Giovanile;
    \item I Centri Federali Territoriali;
    \item L'Attività Scolastica;
    \item La Tutela della salute e della sicurezza;
    \item Le Norme generali per lo svolgimento delle attività giovanili;
    \item La Regolamentazione dei tornei organizzati dalle società.
\end{itemize}
In particolare i valori e le idee su cui si basa l'operato dell'S.G.S si fondano sulla \textbf{Carta dei Diritti dei Ragazzi allo Sport dell'ONU}: 
\begin{enumerate}
    \item Il diritto di divertirsi e giocare;
    \item Il diritto di fare sport;
    \item Il diritto di beneficiare di un ambiente sano;
    \item Il diritto di essere circondato ed allenato da persone competenti;
    \item Il diritto di seguire allenamenti adeguati ai propri ritmi;
    \item Il diritto di misurarsi con giovani che abbiano analoghe possibilità di successo;
    \item Il diritto di partecipare a competizioni adeguate alla propria età;
    \item Il diritto di praticare sport in assoluta sicurezza;
    \item Il diritto di avere i giusti tempi di riposo;
    \item Il diritto di non essere un campione.
\end{enumerate}
e sul \textbf{Decalogo Grassroots Uefa}:
\begin{enumerate}
    \item Il calcio è un gioco per tutti;
    \item Il calcio deve poter essere praticato dovunque;
    \item Il calcio è creatività;
    \item Il calcio è dinamicità;
    \item Il calcio è onestà;
    \item Il calcio è semplicità;
    \item Il calcio deve essere svolto in condizioni sicure;
    \item Il calcio deve essere proposto con attività variabili;
    \item Il calcio è amicizia;
    \item Il calcio è un gioco meraviglioso;
    \item Il calcio è un gioco popolare e nasce dalla strada.
\end{enumerate}

\section{Settore Tecnico}
Il settore tecnico:
\begin{itemize}
    \item Ha la competenza nei rapporti internazionali nelle amterie attinenti la definizione delle regole di gioco, del calcio e delle tecniche di formazione di atleti e tecnici.
    \item Presiede alla formazione, istruzione, qualificazione, abilitazione, aggiornamento, inquadramento e tesseramento dei tecnici autorizzati ad svolgere attività nell'ambito dell'organizzazione federale e societaria.
    \item Organizza, in accordo con il Centro Studi Federale, attività di studio e ricerca in tutti gli aspetti del gioco del calcio e dei fenomeni sociali, culturali, scientifici ed economici ad esso connessi.
    \item Organizza e coordina l'attività medica nell'ambito federale in attuazioe dei regolamenti della F.I.G.C. e inquadra e tesera i medici sociali e gli altri operatori sanitari delle società.
\end{itemize}

\subsection{Tecnici e Abilitazioni}
In particolare, come stabilito dall'Art. 16 e successivi del Regolamento del Settore Tecnico, il Settore Tecnico definisce le qualifiche per i tecnici e le loro abilitazioni:
\begin{itemize}
    \item\textbf{Uefa Pro}: Massima abilitazione ottenibile con corso centrale a Coverciano dopo aver ottenuto la qualifica \emph{Uefa A}. Può :
        \begin{itemize}
            \item Allenare Tutte le Prime Squadre Maschili anche professionistiche
            \item Allenare Tutte le Prime Squadre Feminili anche professionistiche
            \item Allenare Tutte le Squadre di Settore Giovanile (sia agonistiche che di attività di base) compresa la Primavera Maschile o Femminile.
        \end{itemize}
    \item\textbf{Uefa A}: Prima abilitazione professionistica, ottenibile con corso centrale a Coverciano dopo aver ottenuto la qualifica \emph{Uefa B}. Può:
        \begin{itemize}
            \item Allenare Tutte le Prime Squadre Maschili tranne quelle di Serie A e B (per le quali può fare il secondo allenatore o il collaboratore tecnico)
            \item Allenare Tutte le Prime Squadre Feminili anche professionistiche
            \item Allenare Tutte le Squadre di Settore Giovanile (sia agonistiche che di attività di base) compresa la Primavera Maschile o Femminile.
        \end{itemize}
    \item\textbf{Uefa B}: Ottenibile avendo entrambe le qualifiche \emph{Uefa C} e \emph{Licenza D}. Può:
        \begin{itemize}
            \item Allenare Tutte le Prime Squadre Maschili Dilettantistiche (Può fare il secondo allenatore per le squadre di Serie C o il collaboratore tecnico per Serie A-B-C)
            \item Allenare Tutte le Prime Squadre Feminili tranne quelle di Serie A e B (per le quali può fare il secondo allenatore o il collaboratore tecnico)
            \item Allenare Tutte le Squadre di Settore Giovanile Maschili e Femminili (sia agonistiche che di attività di base) esclusa la Primavera Maschile (per la quale può fare il secondo allenatore o il collaboratore tecnico) ma inclusa la Primavera Femminile.
        \end{itemize}
    \item\textbf{Uefa C}: Ottenibile tramite corsi regionali. Può:
        \begin{itemize}
            \item Allenare Tutte le Squadre di Settore Giovanile Maschili e Femminili (sia agonistiche che di attività di base) esclusa la Primavera Maschile ma inclusa la Primavera Femminile.
        \end{itemize}
    \item\textbf{Licenza D}:  Ottenibile tramite corsi regionali. Può:
        \begin{itemize}
            \item Allenare Tutte le Prime Squadre Maschili Dilettantistiche tranne la Serie D.
            \item Allenare Tutte le Prime Squadre Feminili tranne quelle di Serie A e B (per le quali può fare il secondo allenatore o il collaboratore tecnico)
            \item Allenare Tutte le Squadre Juniores Maschili e Femminili (Nazionali, Regionali o Provinciali) esclusa la Primavera Maschile ma inclusa la Primavera Femminile.
        \end{itemize}
\end{itemize}
riassumiamo le possibili abilitazioni nelle seguenti tabelle: \hfill \\

Calcio Maschile a 11  \hfill \\

\begin{table}[!htbp]
    \begin{tabular}{llllllllll}
                       & \multicolumn{1}{c}{\textbf{Serie}}                     & \multicolumn{1}{c}{\textbf{Serie}}              & \multicolumn{1}{c}{\textbf{Serie}}              & \multicolumn{1}{c}{\textbf{Serie}}              & \multicolumn{1}{c}{\textbf{Ecce}}               & \multicolumn{1}{c}{\textbf{Promo}}              & \multicolumn{1}{c}{\textbf{1°}}                 & \multicolumn{1}{c}{\textbf{2°}}                 & \multicolumn{1}{c}{\textbf{3°}}                 \\
    \textbf{}          & \multicolumn{1}{c}{\cellcolor[HTML]{FFFFFF}\textbf{A}} & \multicolumn{1}{c}{\textbf{B}}                  & \multicolumn{1}{c}{\textbf{C}}                  & \multicolumn{1}{c}{\textbf{D}}                  & \multicolumn{1}{c}{\textbf{-llenza}}            & \multicolumn{1}{c}{\textbf{-zione}}             & \multicolumn{1}{c}{\textbf{Cat}}                & \multicolumn{1}{c}{\textbf{Cat}}                & \multicolumn{1}{c}{\textbf{Cat}}                \\
    \textbf{Uefa Pro}  & \cellcolor[HTML]{FE0000}                               & \cellcolor[HTML]{FE0000}                        & \cellcolor[HTML]{FE0000}                        & \cellcolor[HTML]{FE0000}                        & \cellcolor[HTML]{FE0000}                        & \cellcolor[HTML]{FE0000}                        & \cellcolor[HTML]{FE0000}                        & \cellcolor[HTML]{FE0000}                        & \cellcolor[HTML]{FE0000}                        \\
    \textbf{Uefa A}    &                                                        & \cellcolor[HTML]{F8FF00}{\color[HTML]{FFFFFF} } & \cellcolor[HTML]{F8FF00}{\color[HTML]{FFFFFF} } & \cellcolor[HTML]{F8FF00}{\color[HTML]{FFFFFF} } & \cellcolor[HTML]{F8FF00}{\color[HTML]{FFFFFF} } & \cellcolor[HTML]{F8FF00}{\color[HTML]{FFFFFF} } & \cellcolor[HTML]{F8FF00}{\color[HTML]{FFFFFF} } & \cellcolor[HTML]{F8FF00}{\color[HTML]{FFFFFF} } & \cellcolor[HTML]{F8FF00}{\color[HTML]{FFFFFF} } \\
    \textbf{Uefa B}    &                                                        &                                                 &                                                 & \cellcolor[HTML]{34FF34}                        & \cellcolor[HTML]{34FF34}                        & \cellcolor[HTML]{34FF34}                        & \cellcolor[HTML]{34FF34}                        & \cellcolor[HTML]{34FF34}                        & \cellcolor[HTML]{34FF34}                        \\
    \textbf{Uefa C}    &                                                        &                                                 &                                                 &                                                 &                                                 &                                                 &                                                 &                                                 &                                                 \\
    \textbf{Licenza D} &                                                        &                                                 &                                                 &                                                 & \cellcolor[HTML]{C92FD7}                        & \cellcolor[HTML]{C92FD7}                        & \cellcolor[HTML]{C92FD7}                        & \cellcolor[HTML]{C92FD7}                        & \cellcolor[HTML]{C92FD7}                                               
    \end{tabular}
\end{table}

Calcio Femminile a 11 \hfill \\

\begin{table}[!htbp]
    \begin{tabular}{l|l|l|l|l|l|}
    \cline{2-6}
                                             & \multicolumn{1}{c|}{\textbf{Serie A}} & \multicolumn{1}{c|}{\textbf{Serie B}}           & \multicolumn{1}{c|}{\textbf{Serie C}}           & \multicolumn{1}{c|}{\textbf{Eccellenza}}        & \multicolumn{1}{c|}{\textbf{Promozione}}        \\ \hline
    \multicolumn{1}{|l|}{\textbf{Uefa Pro}}  & \cellcolor[HTML]{FE0000}              & \cellcolor[HTML]{FE0000}                        & \cellcolor[HTML]{FE0000}                        & \cellcolor[HTML]{FE0000}                        & \cellcolor[HTML]{FE0000}                        \\ \hline
    \multicolumn{1}{|l|}{\textbf{Uefa A}}    & \cellcolor[HTML]{F8FF00}              & \cellcolor[HTML]{F8FF00}{\color[HTML]{FFFFFF} } & \cellcolor[HTML]{F8FF00}{\color[HTML]{FFFFFF} } & \cellcolor[HTML]{F8FF00}{\color[HTML]{FFFFFF} } & \cellcolor[HTML]{F8FF00}{\color[HTML]{FFFFFF} } \\ \hline
    \multicolumn{1}{|l|}{\textbf{Uefa B}}    &                                       &                                                 & \cellcolor[HTML]{34FF34}                        & \cellcolor[HTML]{34FF34}                        & \cellcolor[HTML]{34FF34}                        \\ \hline
    \multicolumn{1}{|l|}{\textbf{Uefa C}}    &                                       &                                                 &                                                 &                                                 &                                                 \\ \hline
    \multicolumn{1}{|l|}{\textbf{Licenza D}} &                                       &                                                 & \cellcolor[HTML]{C92FD7}                        & \cellcolor[HTML]{C92FD7}                        & \cellcolor[HTML]{C92FD7}                        \\ \hline
    \end{tabular}
\end{table}

Juniores U19 \hfill \\

\begin{table}[!htbp]
    \begin{tabular}{|l|l|l|l|l|}
    \hline
              & \multicolumn{1}{c|}{\begin{tabular}[c]{@{}c@{}}Primavera\\ Maschile\end{tabular}} & \multicolumn{1}{c|}{\begin{tabular}[c]{@{}c@{}}Juniores (N-R-P)\\ Maschile\end{tabular}} & \multicolumn{1}{c|}{\begin{tabular}[c]{@{}c@{}}Primavera\\ Femminile\end{tabular}} & \multicolumn{1}{c|}{\begin{tabular}[c]{@{}c@{}}Juniores (R)\\ Femminile\end{tabular}} \\ \hline
    Uefa Pro  & \cellcolor[HTML]{FE0000}                                                          & \cellcolor[HTML]{FE0000}                                                                 & \cellcolor[HTML]{FE0000}                                                           & \cellcolor[HTML]{FE0000}                                                              \\ \hline
    Uefa A    & \cellcolor[HTML]{F8FF00}                                                          & \cellcolor[HTML]{F8FF00}                                                                 & \cellcolor[HTML]{F8FF00}                                                           & \cellcolor[HTML]{F8FF00}                                                              \\ \hline
    Uefa B    &                                                                                   & \cellcolor[HTML]{34FF34}                                                                 & \cellcolor[HTML]{34FF34}                                                           & \cellcolor[HTML]{34FF34}                                                              \\ \hline
    Uefa C    &                                                                                   & \cellcolor[HTML]{3166FF}                                                                 & \cellcolor[HTML]{3166FF}                                                           & \cellcolor[HTML]{3166FF}                                                              \\ \hline
    Licenza D &                                                                                   & \cellcolor[HTML]{C92FD7}                                                                 & \cellcolor[HTML]{C92FD7}                                                           & \cellcolor[HTML]{C92FD7}                                                              \\ \hline
    \end{tabular}
\end{table}
\hfill \\
Invece per tutte le altre categorie giovanili (Allievi/e, Giovanissimi/e, Esordienti, Primi Calci, Piccoli Amici), indipentenemtente dal livello (Provinciale, Regionale o Nazionale Professionisti), è necessario essere qualificati come \emph{Uefa Pro}, \emph{Uefa A}, \emph{Uefa B}, \emph{Uefa C} mentre non si può allenare possedendo solo la \emph{Licenza D}. \hfill \\
La terza categoria maschile è l'unica categoria a non avere obblighi riguardo ai tecnici.

\subsection{Allenatori dei Portieri}
A livello di allenatori dei portieri, vengono inquadrate le seguenti qualifiche:

\begin{itemize}
    \item\textbf{Uefa GK A}: Massima qualifica ottenibile tramite corso a Coverciano dopo aver ottenuto la qualifica \emph{Uefa B} e la qualifica \emph{Uefa B GK}. Permette di essere tesserati come allenatore dei portieri per ogni squadra sia maschile o femminile di qualunque livello (professionisti e dilettanti) e per ogni squadra di settore giovanile (sia agonistico che attività di base).
    \item\textbf{Allenatore dei Portieri} (\emph{Ad Esaurimento}): Massima qualifica ottenibile prima dell'avvento del \emph{Uefa GK A} tramite corso a Coverciano dopo aver ottenuto la qualifica \emph{Uefa B}. Permette di essere tesserati come allenatore dei portieri per ogni squadra sia maschile o femminile di qualunque livello (professionisti e dilettanti) e per ogni squadra di settore giovanile (sia agonistico che attività di base).
    \item\textbf{Uefa GK B}: Ottenibile tramite corsi regionali dopo aver ottenuto una tra \emph{Uefa B}, \emph{Uefa C} o \emph{Licenza D} (quest'ultima solo in Italia). Permette di essere tesserati come allenatore dei portieri per qualunque squadra maschile dilettantistica, per qualunque squadra femminile escluse serie A e B e per qualunque squadra di settore giovanile, anche professionistica, incluse la primavera maschile e quella feminile.
    \item\textbf{Allenatore dei Portieri Dilettanti e Settore Giovanile}: Ottenibile tramite corsi regionali e non richiede altre qualifiche pregresse. Permette di essere tesserati come allenatore dei portieri per qualunque squadra maschile dilettantistica, per qualunque squadra femminile escluse serie A e B e per qualunque squadra di settore giovanile, anche professionistica, esclusa la primavera maschile ma inclusa quella femminile.
\end{itemize}

Inoltre le Società che svolgono attività di Settore Giovanile o di Base devono tesserare almeno un Allenatore dei Portieri o Allenatore dei Portieri Dilettanti e di Settore Giovanile o GKB o GKA.

\subsection{Calcio a 5}
A livello di Calcio a 5, si hanno le seguenti qualifiche:
\begin{itemize}
    \item\textbf{Allenatori di Calcio a 5 di 1° Livello}: Massima qualifica ottenibile tramite corso a Coverciano dopo aver ottenuto la qualifica di \emph{Uefa B Futsal} o di \emph{Allenatore di Calcio a 5}. Permette di allenare in qualunque squadra di Calcio a 5 sia maschili che femminili (incluse le giovanili).
    \item\textbf{Uefa B Futsal}; Qualifica ottenibile con corsi regionali. Permette di allenare in tutte le squadre di Calcio a 5 esclusa le Serie A e A2 maschili (per le quali si può tesserare come secondo allenatore o collaboratore tecnico) ed incluse le categorie giovanili.
    \item\textbf{Allenatori di Calcio a 5}: Vecchia qualifica di calcio a 5. E' equipollente alla qualifica di \emph{Uefa B Futsal}.
\end{itemize}

Le categorie maschili D e femminili B e C non hanno obblighi in quanto a tecnici richiesti. Inoltre al momento l'allenatore dei portieri di calcio a 5 può essere un tecnico con una delle altre qualifiche di Calcio a 5 ma verrà introdotta nelle prossime stagioni una figura tecnica specifica.

\subsection{Coaching Convention UEFA}
La Uefa riconosce tramite la propria coaching convention, le qualifiche \emph{Uefa Pro}, \emph{Uefa A} e \emph{Uefa B}. Questo significa che vengono automaticamente riconosciute in ognuna delle federazioni nazionali che aderiscono alla coaching convention. Le qualifiche \emph{Uefa A GK}, \emph{Uefa B GK}, \emph{Uefa C} e \emph{Uefa B Futsal}, malgrado siano esplicitamente nominate, non sono indicate come riconosciute automaticamente ma il loro stato viene demandanto alle singole federazioni. \hfill \\
Il settore tecnico, in conformità all'articolo 27 della coaching convention, deve aggiornare ogni tecnico \emph{Uefa Pro}, \emph{Uefa A} e \emph{Uefa B} con corsi di almeno 15 h ogni 3 anni. Nello specifico, vengono organizzati 3 moduli da 5 ore ciascuno che si potranno seguire sia online che in modalità frontale sul territorio.

\subsection{Tecnici e Tesseramento}

Come già indicato nella sezione sul tesseramento, il tesseramento dei tecnici è unico e quindi è vietato svolgere attività per più società, anche con mansioni diverse. \hfill \\
Inoltre i tecnici, per svolgere attività calcistica diversa da quella da tecnico, devono presentare domanda di sospensione all'Albo precisando la natura della nuova società. \hfill \\
Sono esclusi dalla sospensione i tecnici che vogliono svolgere attività quale calciatore o dirigente per la stessa società in cui sono tesserati come tecnici o quelli dilettanti che, non tesserati come tecnici per nessuna società, intendono scolgere attività come calciatori.

\subsection{Disciplina dei Tecnici}

I tecnici sono soggetti alle decisioni del \textbf{Giudice Sportivo} che delibera riguardo infrazioni inerenti all'attività agonistica (es gare di campionato) e della \textbf{Commissione Disciplinare del Settore Tecnico}, la quale prende provvedimenti in caso di:
\begin{itemize}
    \item Tecnici che hanno ottenuto la sospensione ma continuano a svolgere le mansioni derivanti dall'iscrizione all'Albo di Allenatore o di Direttore Tecnico.
    \item Tecnici che violano delle norme deontologiche.
    \item Tecnici che violano l'Art. 40 del Regolamento del Settore Tecnico "Obblighi e Deroghe".
    \item Tecnici che violano l'Art. 41 del Regolamento del Settore Tecnico "Preclusioni e Sanzioni".
\end{itemize}

\end{document}