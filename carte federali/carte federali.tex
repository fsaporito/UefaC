\documentclass[../uefaC.tex]{subfiles}

\subfilecover{Saporito Francesco}{Carte Federali}

\begin{document}

\subfileintro{Carte Federali}

Durata Corso: 2h (Teoria in Aula) \hfill \\
Esame: ? (Assieme a Giustizia Sportiva)

\hfill \\
Le carte federali sono le fonti dell'Ordinamento Federale (Art. 2 VI co. S.F.). In particolare si possono identificare:
\begin{enumerate}
    \item Lo Statuto della Federazione Italiana Gioco Calcio (F.I.G.C.)
    \item Le Norme Organizzative Interne Federali (N.O.I.F.), il Codice di Giustizia Sportiva e le altre disposizioni emanate dal consiglio federale.
    \item Gli Statuti e i Regolamenti delle Leghe, delle Componenti Tecniche, dell'Associazione Italiana Arbitri (A.I.A.), del Settore Tecnico e del Settore Giovanile.
\end{enumerate}

In particolare lo scopo dello studio delle carte federali è molteplice:
\begin{enumerate}
    \item Per esercitare al meglio la propria attività da Allenatore, il cui ruolo e funzione sta cambiando e richiede una sempre maggiore consapevolezza del contesto in cui opera.
    \item Per esercitare al meglio i propri diritti e conoscere i propri doveri.
    \item Per porre le basi per esercitare in futuro carriere differenti nel mondo del calcio.
\end{enumerate}

\subsection{Organizzazione FIGC}



\subsection{Comunicati n. 1 SGS}

\subsection{N.O.I.F e Giustizia Sportiva in Ambito Giovanile}

\end{document}