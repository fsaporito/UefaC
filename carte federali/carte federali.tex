\documentclass[../uefaC.tex]{subfiles}

\subfilecover{Saporito Francesco}{Carte Federali}

\begin{document}

\subfileintro{Carte Federali}

\hfill \\
Durata Corso: 2h (Teoria in Aula) \hfill \\
Esame: ? (Assieme a Giustizia Sportiva)

\hfill \\
\section{Introduzione}
Le carte federali sono le fonti dell'Ordinamento Federale (Art. 2 VI co. S.F.). In particolare si possono identificare:
\begin{enumerate}
    \item Lo Statuto della Federazione Italiana Gioco Calcio (F.I.G.C.)
    \item Le Norme Organizzative Interne Federali (N.O.I.F.), il Codice di Giustizia Sportiva e le altre disposizioni emanate dal consiglio federale.
    \item Gli Statuti e i Regolamenti delle Leghe, delle Componenti Tecniche, dell'Associazione Italiana Arbitri (A.I.A.), del Settore Tecnico e del Settore Giovanile.
\end{enumerate}
\hfill \\
In particolare lo scopo dello studio delle carte federali è molteplice:
\begin{enumerate}
    \item Per esercitare al meglio la propria attività da Allenatore, il cui ruolo e funzione sta cambiando e richiede una sempre maggiore consapevolezza del contesto in cui opera.
    \item Per esercitare al meglio i propri diritti e conoscere i propri doveri.
    \item Per porre le basi per esercitare in futuro carriere differenti nel mondo del calcio.
\end{enumerate}

\section{Organizzazione FIGC}

La Federazione Italiana Gioco Calcio (F.I.G.C.) svolge le proprie funzioni, in piena autonomia tecnica e di gestione, in armonia con le deliberazioni e gli indirizzi della FIFA (Fédération Internationale de Football Association), della UEFA (Union of European Football Associations), del CIO (Comitato Olimpico Internazionale) e del CONI (Comitato Olimpico Nazionale Italiano). \hfill \\
L'organizzazione centrale della FIGC è composta dai seguenti ruoli / organi:
\begin{enumerate}
    \item L'Assemblea
    \item Il Presidente
    \item I Vice-Presidenti
    \item Il Comitato di Presidenza
    \item IL Consiglio Federale
    \item Il Direttore generale
    \item Il Collegio di dei Revisori dei Conti
\end{enumerate}
\hfill \\
La FIGC costituisce inoltre una propria organizzazione periferica secondo norme approvate dal Consiglio Federale. Fino a tale costituzione, i Presidenti dei Comitati Regionali e i delegati Provinciali della Lega Nazionale Dilettanti (LND) esercitano le funzioni rappresentative della FIGC ad essi delegate dal Consiglio Federale o dal Presidente Federale nei rapporti con le rispettive strutture periferiche del CONI, nonché in eventuali altri compiti di rappresentanza federale nel territorio di competenza, fatta salva la eventuale diversa delega. \hfill \\
Sono parte dell'Assemblea le seguenti componenti:
\begin{enumerate}
    \item Lega Nazionale Professionisti Serie A
    \item Lega Nazionale Professionisti Serie B
    \item Lega Calcio Professionistico
    \item Lega Nazionale Dilettanti (LND)
    \item Associazione Italiana Calciatori (AIC)
    \item Associazione Italiana Allenatori (AIAC)
    \item Associazione Italiana Arbitri (AIA)
\end{enumerate}
La FIGC è poi composta da due settori:
\begin{enumerate}
    \item{Settore Giovanile e Scolastico (SGS)} Promuove, discliplina e organizza l'attività dei giovani calciatori, con finalità tecniche, didattiche e sociali. Definisce in particolare l'attività calcistica giovanile in 3 diverse fasi:
    \begin{itemize}
        \item{Attività di Base} (Età dai 5 ai 12 anni)
        \item{Attività Agonistica} (Età dai 12 ai 16 anni)
        \item{Attività Scolastica}
    \end{itemize}
    \item{Settore Tecnico} E' l'organo di servizio della FIGC incaricato a norma dell'art. 14 dello statuto di svolgere attività di studio e qualificazione per la diffusione e il miglioramento della tecnica del gioco del calcio.
\end{enumerate}
\hfill \\
e da due divisioni:
\begin{enumerate}
    \item{Calcio Femminile} Si occupa della gestione ed organizzazione di tutti i campionait femminli. In particolare:
    \begin{itemize}
        \item{FIGC}: Si occupa della strategia di promozione e sviluppo del settore, dell'attività delle nazionali (Nazionale A, U23, U19, U17, U16) e dell'organizzazione dei campionati di Serie A e Serie B, del campionato Primavera e della Coppa Italia.
        \item{LND}: Si occupa dell'organizzazione del Campionato Nazionale di Serie C e dei Campionati Regionali di Eccellenza Femminile, Promozione Femminile e della Coppa Italia Regionale.
    \end{itemize}
    \item{Calcio a 5}: Si Occupa della promozione e dello sviluppo del settore, oltre all'organizzazione dei campionati nazionali (Serie A, Serie A2, Serie B, Serie A femminile, Serie A2 femminile), regionali e provinciali (Serie C1, Serie C2, Serie D), Campionato Nazionale U21 (Maschile), Campionato Nazionale U19 (sia maschile che femminile) e la Coppa Italia (sia maschile che femminile).
\end{enumerate}

\section{N.O.I.F}

\section{Settore Giovanile Scolastico (S.G.S)}

\section{Settore Tecnico}

\end{document}