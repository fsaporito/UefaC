\documentclass[../uefaC.tex]{subfiles}

\subfilecover{Saporito Francesco}{Diritti dei Bambini}

\begin{document}

\subfileintro{Diritti dei Bambini (Parte di Tecnica e Tattica Sportiva)}

I diritti dei Bambini in senso lato e dei giovani calciatori in senso stretto, vengono indicati in tre documenti specifici: \hfill \\

\begin{enumerate}
    \item{Carta dei Diritti dei Ragazzi allo Sport}
    \item{Carta Grassroots Uefa (Uefa Grassroots Charter)}
    \item{Comunicato N. 1 del Settore Giovanile Scolastico}
\end{enumerate}
\hfill \\

\section{Carta dei Diritti dei Ragazzi allo Sport}

\begin{enumerate}
    \item{\textbf{Il Diritto di Divertirsi e Giocare}} \\L'impronta deve essere mirata al divertimento e all'aspetto ludico piuttosto che alla vittoria.
    \item{\textbf{Il Diritto di Fare Sport}} \\ Ognuno deve poter fare sport indipendentemente dai suoi risultati o attitudine.
    \item{\textbf{Il Diritto di Beneficiare di un Ambiente Sano}} \\Sia dal punto di vista della sicurezza "fisico/sanitaria" che da quello emotivo. Ad esempio i giovani devono poter sperimentare senza avere paura di sbagliare o di essere rimproverati.
    \item{\textbf{Il Diritto di Essere Circondato e Allenato da Persone Competenti}} \\Se non si è competenti si rischia di fare danni irreparabili sia fisici che emotivo-psicologici ai bambini.
    \item{\textbf{Il Diritto di Seguire Allenamenti adeguati ai suoi Ritmi}} \\Gli allenamenti devono essere mirati alle effettive capacità del bambino e non essere messi ad un ideale livello che non rappresenta la realtà.
    \item{\textbf{Il Diritto di Misurarsi con Giovani con le stesse probabilità di Successo}} \\La competizione è importante per la crescità e la maturazione ma deve essere impostata in modo che tutti ne possano trarre vantaggio e divertisti nel farla.
    \item{\textbf{Il Diritto di Partecipare a competizioni adeguate alla sua Età}} \\Non bisogna bruciare le tappe ma permettere ad ognuno di seguire il naturale processo di apprendimento motorio e crescità fisica.
    \item{\textbf{Il Diritto di Praticare Sport in Assoluta Sicurezza}} \\La sicurezza sia fisica che emotiva dei bambini è fondamentale affinchè possano concentrarsi sullo sport senza altre preoccupazioni.
    \item{\textbf{Il Diritto di avere i giusti Tempi di Riposo}} \\Affinchè gli allenamenti siano da un lato efficaci e dall'altro non lesivi, il riposo è fondamentale.
    \item{\textbf{Il Diritto di non Essere un Campione}} \\Ognuno deve essere spinto a migliorarsi e a crescere rispetto al suo passato curando l'aspetto formativo, non per arrivare ad un livello target per la vittoria nel confronto con altri bambini/squadre.
\end{enumerate}
\hfill \\

In particolare, bisogna posizionare il bambino al centro dell'attività, portandolo alla continua crescità e miglioramento tramite il divertimento, le emozioni positive e il continuo affrontare problemi motori diversi a cui non viene data una soluzione dall'esterno (modalità direttiva), ma deve essere condotto il bambino a trovare in prima persona le proprie soluzioni.

\section{Carta Grassroots Uefa}
Questo documento è stato redatto dalla Uefa nel 2004 in base alle seguenti considerazioni: \hfill \\
\begin{itemize}
    \item{La qualità futura del calcio si trova in gran parte nel calcio di base.}
    \item{Le federazioni nazionali hanno sviluppato o sono interessate a sviluppare progetti nazionali per il calcio di base}
    \item{La Uefa Mette in atto il Programma Uefa per il Calcio di Base (Uefa GRASSROOT PROGRAM), basato sui programmi del calcio di base sviluppati dalle Federazioni Nazionali e attraverso il quale la Uefa intende promuovere, tutelare e sviluppare il calcio di base in tutte le sue forme.}
    \item{L'Uefa desidera monitorare e appoggiare la migliore condotta delle Federazioni Nazionali ed incoraggiarle ad aspirare alla UEfa come punto di riferimento.}
\end{itemize}
\hfill \\
In particolare sono rilevanti i seguenti paragrafi \textcolor{red}{TODO}


\section{Comunicato N. 1 del Settore Giovanile Scolastico}

Viene pubblicato annualmente e regolamenta tutta l'attività del settore giovanile. Richiama i diritti dei bambini e degli atleti e specifica per ogni età e catgoria la struttura dei campionati, le dimensioni del campo e dei palloni e le modalità di gioco. \textcolor{red}{TODO}


\end{document}
