\documentclass[../uefaC.tex]{subfiles}

\subfilecover{Saporito Francesco}{Tecnica del Portiere}

\begin{document}

\subfileintro{Tecnica del Portiere}

Durata Corso: 6h (Teoria in Aula) \hfill \\
Esame: Nessuno (Sottointesto in Tecnica e Tattica Calcistica)

\section{Storia del Ruolo}
Il portiere è il giocatore preposto alla difesa della porta ed è esplicitamente nominato dal regolamento come unico giocatore sempre obbligatorio (non si può giocare senza portiere). La regolamentazione del ruolo si è però molto evoluta negli anni, sempre allo scopo di velocizzare il gioco e di ridurre le perdite di tempo e i tentativi di comportamento antisportivo ad esso associati. \hfill \\
Inizialmente il portiere ha la facoltà di toccare il pallone con le mani in tutto il terreno di gioco ed è solo nel 1912 che viene limitato il tocco di mano alla sola area di rigore. \hfill \\
Dato che però il portiere non può essere contrastato dagli avversari quando ha la palla in mano, per evitare che egli tenga il pallone per un tempo indefinito, nel 1983 è stata introdotto il divieto per il portiere di effetturae più di 4 passi con il pallone anche se non stabilendo le modalità ne i tempi per effettuarli la finalità non fu inizialmente raggiunta. Soltanto nel 2000 fu cambiata questa regola andando ad introdurre un tempo massimo di $6$ secondi in cui il portiere può tenere in mano il pallone. \hfill \\
Sempre in questa ottica nel 1992 è stato introdotto il divieto per il portiere di prendere con le mani la palla proveniente da un retropassaggio volontario, modifica introdotta dato che spesso però le squadre passavano la palla al proprio portiere per perdere tempo ed impedire agli avversari di giocarlo. \hfill \\
E' infine del 2019 il cambio della regola sulla rimessa dal fondo: prima tutti i giocatori diversi dal tiratore dovevano aspettare fuori dall'area di rigore per ricevere il pallone mentre con questo cambiamento si introduce la possibilità per i compagni di squadra di ricevere il pallone direttamente all'interno dell'area di rigore. \hfill \\ 
\hfill \\
C'è stata dunque un'evidente evoluzione del ruolo, determinata sia dalle modifiche regolamentari che dall'evoluzione della tattica calcistica:
\begin{itemize}
    \item Inizialmente il Portiere ha il ruolo di sola difesa della porta.
    \item Con la sparizione della figura del libero, al portiere vengono affidati anche compiti di difesa dello spazio andando a creare la figura ibrida cosiddetta \emph{sweeper-keeper}(Portiere-Libero).
    \item Il portiere moderno invece si considera anche un regista a tutti gli effetti per la squadra e partecipa alla costruzione del gioco come un qualunque altro giocatore di movimento.
\end{itemize}
Un esempio di questa evoluzione lo si può vedere nel rinvio dal fondo dove inizialmente era il libero ad occuparsene per poi lasciare spazio al portiere che cercava solamente di fare un rinvio più lungo possibile. Il portiere moderno invece cerca una soluzione mirata al mantenimento del possesso palla, andando ad effettuare rinvii corti e a proporsi subito dopo come appoggio o sostegno come un qualunque altro giocatore.


\section{Modello Prestativo del Portiere}
Il modello prestativo del portiere è fondamentale per impostare l’allenamento: studiare la prestazione del portiere significa analizzare ciò che il portiere fa in partita, sia in fase offensiva che difensiva. \hfill \\
In particolare si è visto che, considerando come campione statistico i portieri nazionali in un mondiale, si ha:
\begin{itemize}
    \item I Portieri effettuano azioni difensive per il 25 - 30 \% del tempo. (Difesa della porta e Dello Spazio).
    \item I Portieri effettuano azioni offensive per il 70 - 75 \% del tempo. (Gestione podalica del pallone, rimessa dal fondo e punizioni, transizioni positive).
\end{itemize}
Le statistiche sono simili sia per i dilettanti che per il settore giovanile. \hfill \\
Si ha inoltre che in media un portiere percorre 4 km in una partita, di cui la maggior parte vengono effettuati camminando o correndo a bassa velocità. \hfill \\
A livello invece di tempi di intervento, si ha che mediamente un intervento dura circa 2-3 secondi (es tuffo, uscita alta/bassa, ...) e tra un intervento e l'altro il portiere ha circa 3 minuti di recupero. \hfill \\
Viene dunque usato prevalentemente il meccanismo anaerobico alattacido nei momenti attivi, accompagnato da un forte uso del meccanismo aerobico a bassa intensità.

\section{Differenze tra il Portiere e i Giocatori di Movimento}
Malgrado l'evoluzione del portiere lo porti sempre più vicino ad un giocatore di movimento, rimangono comunque delle differenze fondamentali:
\begin{itemize}
    \item Regolamentari:
        \begin{itemize}
            \item Il portiere deve avere una maglia diversa.
            \item Ogni squadra deve sempre avere un portiere.
            \item Il portiere può toccare la palla con le mani nella propria area di rigore.
        \end{itemize}
    \item Bioenergetiche:
        \begin{itemize}
            \item Il portiere come visto nella sezione sul modello prestativo utilizza il meccanismo anaerobico alattacido negli interventi mentre il meccanismo aerobico nel resto del tempo.
            \item I giocatori di movimento invece lavorano su tutti e 3 i meccanismi energetici andando a compiere sforzi di diversa intensità. In particolare un giocatore di serie A in media percorre 10 km di cui (ovviamente variabili in base anche al ruolo):
                \begin{itemize}
                    \item circa 37 \% camminando
                    \item circa 44 \% corsa leggera
                    \item circa 13 \% elevata velocità 
                    \item circa 6 \% scatto
                \end{itemize}
            \item FC media del portiere è di circa 75-80 \% della Max, influenzata soprattuto dallo stato tensivo mentre quella di un giocatore di movimento si aggira attorno agli 85-90 \%.
        \end{itemize}
    \item Motorie:
        \begin{itemize}
            \item Il portiere ha generalmente un livello maggiore di flessibilità e coordinazione.
            \item Il portiere sviluppa fortemente le capacità coordinative di reazione e di anticipazione motoria.
            \item Il portiere è usualmente più veloce nei movimenti brevi e in quelli ampi.
        \end{itemize}
    \item Antropometriche:
        \begin{itemize}
            \item Il portiere è di solito di altezza superiore rispetto ai giocatori di movimento.
            \item Si favoreggiano portieri che abbiano spalle larghe e un'estesa apertura delle braccia.
            \item E' poi importante per il portiere avere mani grandi e ampie per una migliore gestione della palla.
        \end{itemize}
    \item Psicologiche:
        \begin{itemize}
            \item Il portiere è molte volte estraneo al gioco, il che richiede di dover mantenere una concentrazione e una disciplina maggiore.
            \item Il portiere ha molte responsabilità dato che ad un suo errore quasi sempre scaturisce un'occasione da goal per gli avversari. Deve quindi essere sicuro, caparbio e deve avere fiducia in se stesso.
            \item Il portiere ha una posizione privilegiata in quanto riesce a vedere una porzione maggiore di campo rispetto ai difensori. Deve quindi saper comunicare, essere estroverso e in grado di diventare il leader della propria difesa.
        \end{itemize}
\end{itemize}

\section{L'Allenatore dei Portieri}
\textcolor{red}{TODO}

\section{Il Giovane Portiere}
Storicamente nel settore giovanile, soprattuto nei più piccoli, i portieri venivano sempre scelti rispetto a parametri errati:
\begin{itemize}
    \item meno dotati tecnicamente con il pallone.
    \item meno propensione alla corsa.
    \item "sovrappeso"
\end{itemize}
Bisogna invece analizzare:
\begin{itemize}
    \item Propensione al contatto con il terreno
    \item Abiità Motorie
    \item Coraggio e Determinazione 
    \item Statura (in prospettiva, assieme a quella dei genitori/fratelli/sorelle).
\end{itemize}
Bisogna poi lavorare sfruttando sempre l'aspetto ludico per favorire l'apprendimento, andando a curare la tecnica di base con poche ma mirate correzioni. E' poi fondamentale non escludere il bambino dal gruppo squadra per farlo specializzare solo nel ruolo da portiere, ma deve continuare ad allenarsi con i propri compagni sia per condividere l'allenamento con la sua squadra che per migliorare la tecnica da giocatore e per conoscere lo sport che pratica.

\section{Fondamentali Tecnici del Portiere}
\textcolor{red}{TODO}


\end{document}