\documentclass[../uefaC.tex]{subfiles}

\subfilecover{Saporito Francesco}{Regolamento di Gioco}

\begin{document}

\subfileintro{Regolamento di Gioco}

Durata Corso: 6h (Teoria in Aula) \hfill \\
Esame: Esame a Crocette

\section{Introduzione}
Il regolamento del gioco del calcio è composto da 17 regole + 1 (il buon senso), le quali stabiliscono tutti i dettagli necessari per lo svolgimento del gioco. IN particolare queste regole trattano dei \textbf{Principi} ispiratori del sistema e di ciò che avviene in campo. L'arbitro è dunque la figura che, utilizzando il buon senso, conduce la gara nel rispetto delle regole e dei principi del gioco del calcio. In particolare l'arbitro si deve basare sui 3 seguenti pilastri:
\begin{enumerate}
    \item Regolamento
    \item Preparazione Atletica
    \item Personalità
\end{enumerate}

\section{Regola 1: Il Terreno di Gioco}
Stabilisce le dimensioni e la struttura del campo di gioco. In particolare definisce che il campo deve essere di materiale naturale o artificiale o ibrida (in caso sia artificiale o ibrido, deve essere di colore verde). \hfill \\
Il campo deve essere di forma \emph{rettangolare} con le due linee di delimitazione sul lato lungo denominate \emph{linee laterali} e quelle sul alto corto denominate \emph{linee di porta}. Il campo viene diviso in due metà dalla \emph{linea mediana}, parallela alle linee di porta e al centro di cui viene tracciato il cerchio di centrocampo di raggio \emph{9.15 m} e di origine corrispondente al punto mediano della linea. \hfill \\
Le linee devono essere possibilmente tutte uguali in larghezza, che deve essere al massimo di 12 cm. Le linee di porta devono avere la stessa larghezza dei pali e della traversa.
I campi devono avere le seguenti dimensioni minime e massime, range eventualmente restringibile dalle regole del campionato:
Larghezza: Min 45 m, Max 90 m
Lunghezza: Min 90 m, Max 120 m

\subsection{Porta}
La porta deve essere larga \emph{7.32 m} e alta \emph{2.44 m} con i pali e la traversa di colore bianco e con la stessa larghezza, che deve essere al massimo di 12 cm. Se si rompe la trversa, il gioco dovrà essere sospeso e, in caso di impossibiità della sua riparazione, la gara deve essere sospesa.

\subsection{Area di Porta}
L'area di porta è formata da due linee tracciate perpendicolarmente alla linea di porta, a distanza \emph{5.5 m} dai pali e di lunghezza \emph{5.5 m}. Queste due linee sono congiunte da una linea parallela all'area di porta (che geometricamente avrà una lunghezza di $ 5.5 + 5.5 + 7.32 \; = \; 18.32$).

\subsection{Area di Rigore}
L'area di rigore è formata da due linee tracciate perpendicolarmente alla linea di porta, a distanza \emph{16.5 m} dai pali e di lunghezza \emph{16.5 m}. Queste due linee sono congiunte da una linea parallela all'area di porta (che geometricamente avrà una lunghezza di $ 16.5 + 16.5 + 7.32 \; = \; 40.32$).

\subsection{Area D'angolo}
L'area d'agolo è delimitata da un quarto di circonferenza di un metro di raggio tracciatao all'interno del terreno di gioco da ciascuna bandierina d'angolo.

\subsection{Impraticabilità del Campo}
Il giudizio sull'impraticabilità del campo, per intemperie o per altro motivo, è di esclusiva competenza dell'arbitro il quale alla presenza dei capitani ne effettua l'accertamento. I capitani comunque possono, in forma scritta, presentare delle riserve all'arbitro sulla sua regolarità o praticabilità.


\section{Regola 2: Il Pallone}
Il pallone deve essere di forma sferica e di materiale approvato. Deve avere una circonfernza tra i 68 e i 70 cm, un peso tra i 410 e i 450 g e una pressione tra le 0.6 - 1.1 atmosfere.

\section{Regola 3: I Calciatori}
Stabilisce il numero di giocatori massimo (11) e minimo (7) che possono partecipare attivamente ad una gara, il numero di sostituzoini possibili e le modalità in cui esse debbano essere fatte. In particolare deve essere sempre essere presente un giocatore identificato come portiere.

\section{Regola 4: L'equipaggiamento dei Calciatori}
Stabilisce l'equipaggiamento obbligatorio e vietato per i calciatori. In particolare sono vietati tutti gli accessori di gioielleria o bigiotteria (collane, anelli, orologi, ..), sono obbligatori i parastinchi e le due squadre (e i due portieri) devono facilmente essere identificabili.

\section{Regola 5: L'arbitro}
Definisce i compiti e lo status dell'arbitro. In particolare viene stabilito che l'arbitro ha tutta l'autorità necessaria a far osservare le regole del gioco del calcio e ha discrezionalità di assumere azioni appropriate nel quadro delle regole di gioco. \hfill \\
Le decisioni dell'arbitro sono inappellabili e devono essere rispettate da tutti. L'arbitro può eventualmente cambiare decisione fintanto che il gioco non sia ripreso.

\section{Regola 6: Gli altri Assistenti di Gara}
Definisce chi sono gli altri assistenti di gara, i loro compiti e mansioni.

\section{Regola 7: Durata della Gara}
Definisce il tempo di durata della gara (due tempi da 45 minuti), la durata dell'intervallo (15 minuti) e l'eventuale recupero.

\section{Regola 8: Inizio e Ripresa del Gioco}
Stabilisce come viene iniziato il gioco e come riprende dopo uno stop determinato dall'arbitro o dall'uscità della palla o dalla segnatura di un goal. IN particolare il gioco inizia (o riprende) solo se la palla è in movimento ed è stata evidentemente calciata. Si può segnare su una ripresa del gioco (es da calcio di inizio) ma non si può effettuare un autogoal (es da calcio di punizione nella propria porta deriva un calcio d'angolo per gli avversari). \hfill \\
Definisce poi i modi in cui la ripresa del gioco arrivi direttamente dall'arbitro ovvero la palla viene data al giocatore più vicino della squadra che aveva la palla prima dell'interruzione)

\section{Regola 9: Pallone in Gioco e non in Gioco}
Stabilisce quando il pallone è o no in gioco. In particolare deve superare interamente le linee di porta o laterali per non essere più in gioco.

\section{Regola 10: Esito di Una Gara}
Stabilisce in che occasione viene segnato un goal e chi è la squadra vincitrice. Stabilisce poi come procedere per determinare la squadra vincente in caso di pareggio se il regolamento del campionato imponga la determinazione di una vincente.

\section{Regola 11: Il Fuorigioco}
Un giocatore si trova in posizione di fuorigioco se una qualsiasi parte della testa, del corpo o dei piedi è nella metà campo avversaria e se una qualsiasi parte della testa, del corpo o dei piedi è più vicina all'area di porta avversaria sia rispetto al pallone che al penultimo avversario. \hfill \\
\textbf{Essere in fuorigioco non è di per se un'infrazione} ma lo diventa quando il giocatore viene coinvolto in modo attivo, ad esempio, giocando o toccando il pallone passato da un compagno o interferendo con un avversario. \hfill \\
Non si ha infrazione se il giocatore in fuorigioco riceve palla da un calcio di rinvio, da un calcio d'angolo o da una rimessa laterale.

\section{Regola 12: Falli e Scorrettezze}
Definisce come e quando assegnare un calcio di punizione diretto, un calcio di punizione indiretto o un calcio di rigore, che possono essere assegnati solo per infrazioni commesse quando il pallone è in gioco. \hfill \\
In particolare si assegna un calcio di punizione diretto se si svolge una delle seguenti infrazioni, in modo negligente, imprudente o con vigoria sproporzionata:
\begin{itemize}
    \item Caricare
    \item Saltare Addosso
    \item Dare o Tentare di Dare un calcio 
    \item Spingere 
    \item Colpire o Tentare di Colpire 
    \item Effetturae un tackle o un constrasto
    \item Sgambettare o Tentare di Sgambettare
\end{itemize}
Viene altresì assegnato in caso di una delle seguenti infrazioni:
\begin{itemize}
    \item Fallo di Mano
    \item Trattenuta di Un Avversario
    \item Ostacolamento di Un Avversario con Contatto
    \item Sputa o Morde qualcuno iscritto nell'elenco squadre o un ufficiale di gara 
    \item Lancia un oggetto contro il pallone, un avversario o un ufficiale di gara 
\end{itemize}
In tutti gli altri casi viene assegnato un calcio di punizione indiretto.\hfill \\
Questa regola inoltre stabilisce in che modo punire i calciatori, come gestire il vantaggio e le infrazioni di dirigenti e tecnici.

\section{Regola 13: I Calci di Punizione}
Questa regola definisce come vanno svolti i calci di punizione diretti e indiretti (dati come sanzione alle infrazioni nella regola di cui sopra). In particolare il pallone risulterà in gioco se è in movimento e viene calciato chiaramente, e fino a che non sarà in gioco tutti gli avversari dovranno mantenere una distanza in 9.15 m.


\section{Regola 14: I Calci di Rigore}
Il calcio di rigore viene assegnato quando un calciatore commette una delle infrazioni punibili con un calcio di punizione diretto, all'intero dell'area di rigore. Il pallone deve essere fermo nel punto del calcio di rigore e il giocatore incaricato di eseruire il calcio di rigore deve essere chiaramente identificato. \hfill \\
Il portiere deve essere sulla linea di porta con almeno un piede, senza toccare ne pali ne traverse e facendo fronte al giocatore che tira, fino al momento in cui il pallone non sia stato calciato. \hfill \\
Tutti gli altri calciatori devono stare almeno a 9.15 m dal punto del calcio di rigore, dietro al punto del calcio di rigore, fuori dall'area di rigore e dentro il terreno di gioco.
\section{Regola 15: Rimessa Laterale}
Stabilisce come deve essere effettuata una rimessa laterale.

\section{Regola 16: Calcio di Rinvio}
Stabilisce le modalità in cui deve essere effettuato il calcio di rinvio.

\section{Regola 17: Calcio D'Angolo}
Stabilisce le modalità in cui deve essere effettuato il calcio d'angolo.

\end{document}