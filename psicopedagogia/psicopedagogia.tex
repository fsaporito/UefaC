\documentclass[../uefaC.tex]{subfiles}

\subfilecover{Saporito Francesco}{Psicopedagogia}

\begin{document}

\subfileintro{Psicopedagogia}

Durata Corso: 6h (Teoria in Aula) \hfill \\
Esame: Scritto (20 Crocette) + Orale (solo se scritto con voto < 18)

\textcolor{red}{TODO}

\section{La psicologia dello Sport e SGS}

La psicologia è una giovane disciplina nata come specifica area teorico-applicativa della Psicologia, di cui usa i modelli teorici e gli strumenti, associati a diversi contributi dalla Scienza dell'Allenamento e della Sociologia adattandoli allo specifico campo applicativo.

\section{Il Coach}
L'Allenatore (Coach) ha in realtà molteplici ruoli che tutti assieme vanno a definirne le competenze:
\begin{itemize}
    \item Facilitatore di Apprendimento
    \item Leader
    \item Educatore
    \item Formatore
    \item Tecnico
    \item Organizzatore
\end{itemize}

\subsection{Leader}
Il Coach è il leader formale della propria squadra, ovvero viene scelto all'esterno del gruppo e non emerge autonomamente da esso. Deve dunque essere autoritario 

Fare l'allenatore è anche essere educatore (ex-ducere: tirare fuori ciò che si ha dentro) ovvero valorizzare al massimo l'atleta accedendo il più possibile da quello che ha dentro, creando le condizioni per porlo in uno stato \emph{positivo} di benessere (sia mentale che fisico) che lo stimoli a raggiungere il suo livello soggettivo di eccellenza. Questo comporta quindi andare a far sfidare l'atleta anche contro se stesso, per portarlo oggi ad essere migliore dell'atleta di ieri e peggiore di quello di domani. \hfill \\

L'allenatore deve quindi:
\begin{itemize}
    \item \textbf{Sapere}: Avere \underline{conoscenza} della materia che insegna.
    \item \textbf{Saper Fare}: Essere in grado di dimostrare i movimenti e la tecnica (abilità personale)
    \item \textbf{Saper Far Fare}: Essere in grado di trasmettere il proprio sapere agli atleti, tramite la metodologia di allenamento, la metodologia didattica e la capacità di osservare e correggere i movimenti.
\end{itemize}
E' di particolare importanza inoltre per allenare conoscere come avviene l'apprendimento, inteso come il processo di acquisizione e modificazione di capacità e abilità comportamentali nel corso delle esperienze. Esistono infatti più approcci: \hfill \\
\hfill \\ 
\underline{Approccio Comportamentista}: L'apprendimento è un cambiamento di comportamento. Esiste dunque l'apprendimento quando l'individuo da una risposta corretta ad uno stimolo. I comportamenti sono dunque determinati e modulati sia dalle condizioni ambientali che dai rinforzi (positivi o negativi) che il contesto fornisce a determinate azioni. \hfill \\ 
\hfill \\ 
\underline{Approccio Cognitivista}: L'apprendimento è l'insieme delle attività e dei processi interni inerenti all'acquisizione delle conoscenze, alle informazioni, alla memoria, al pensiero, alla creatività, alla percezione come pure alla comprensione e alla risoluzione dei problemi. La chiave dell'apprendimento risiede nella modalità in cui gli stimoli esterni vengono elaborati internamente. \hfill \\ 
\hfill \\ 
\underline{Approccio Sociale}: L'uomo è un essere senziente che non è guidato ne da forze interne ne da forze ambientali, ma è prima di tutto un essere senziente in grado di modificare il proprio comportamento in base a:
\begin{itemize}
    \item \underline{Feedback} ricevuti in esperienze simili già vissute (Positivi o Negativi).
    \item \underline{Aspettative}, caratteristiche di personalità, sistema di valori posseduto.
    \item \underline{Osservazioni} di successi e fallimenti dei propri simili.
\end{itemize}
In particolare con il termine \underline{Modeling} si intende la capacità di aprpendere senza dover provare in prima persona ogni esperienza ovvero osservando un modello si può apprendere un gesto tecnico o un comportamento corretto da tenere in determinate situazioni. \hfill \\ 
\hfill \\ 
\underline{Approccio Neuroscienze}: Il cervello è \emph{Plastico}, ovvero ha la capacità di cambiare nel corso della vita. Questo implica che è in grado di adattasti alle circostanze e all'apprendimento. In particolare gli stimoli per la nostra attenzione e per i nostri processi di memorizzazione sono:
\begin{itemize}
    \item \underline{Ripetizione}: coinvolgendo la memoria procedurale, ripetere un determinato gesto ci permette di affinarne le modalità di esecuzione (il saper fare).
    \item \underline{Movimento}: i processi motori svolgono un ruolo fondamentale in varie operazioni cognitive, rinforzando o costruendo nuove sinergie o connessioni.
    \item \underline{Emozioni}: Contribuiscono a rendere più efficace l'apprendimento. n particolare le dividiamo in \emph{emozioni piacevoli}, che attivano meccanismi di ricerca efficaci al miglioramento della persona e portano la persona a voler tornare in quella situazione dove in passato ha provato quell'emozione, e nelle \emph{emozioni spiacevoli}, che attivano la memoria dell'alert portando la persona a tendere di evitare le esperienze legate a tali emozioni.
\end{itemize}
\hfill \\ 


\section{L'Ambiente}

\section{Il Giocatore}

\section{Il Gioco}

\end{document}