\documentclass[../uefaC.tex]{subfiles}

\subfilecover{Saporito Francesco}{Psicopedagogia}

\begin{document}

\subfileintro{Psicopedagogia}

Durata Corso: 6h (Teoria in Aula) \hfill \\
Esame: Scritto (20 Crocette) + Orale (solo se scritto con voto < 18)

\textcolor{red}{TODO}

\section{La psicologia dello Sport e SGS}

La psicologia è una giovane disciplina nata come specifica area teorico-applicativa della Psicologia, di cui usa i modelli teorici e gli strumenti, associati a diversi contributi dalla Scienza dell'Allenamento e della Sociologia adattandoli allo specifico campo applicativo.

\section{Il Coach}
L'Allenatore (Coach) ha in realtà molteplici ruoli che tutti assieme vanno a definirne le competenze:
\begin{itemize}
    \item Facilitatore di Apprendimento
    \item Leader
    \item Educatore
    \item Formatore
    \item Tecnico
    \item Organizzatore
\end{itemize}

\subsection{Leader}
Il Coach è il leader formale della propria squadra, ovvero viene scelto all'esterno del gruppo e non emerge autonomamente da esso. Deve dunque essere autoritario 

Fare l'allenatore è anche essere educatore (ex-ducere: tirare fuori ciò che si ha dentro) ovvero valorizzare al massimo l'atleta accedendo il più possibile da quello che ha dentro, creando le condizioni per porlo in uno stato \emph{positivo} di benessere (sia mentale che fisico) che lo stimoli a raggiungere il suo livello soggettivo di eccellenza. \hfill \\
\section{L'Ambiente}

\section{Il Giocatore}

\section{Il Gioco}

\end{document}