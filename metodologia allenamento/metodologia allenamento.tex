\documentclass[../uefaC.tex]{subfiles}

\subfilecover{Saporito Francesco}{Metodologia dell'Allenamento}

\begin{document}

\subfileintro{Metodologia dell'Allenamento}

Durata Corso: 18h (14h Teoria in Aula, 4h Pratica sul Campo) \hfill \\
Esame: Tesina

\section{Maturazione \& Crescita}

La prima cosa che bisogna fare quando ci si affaccia all'allenamento di un giocatore o di una squadra del settore giovanile è capire "chi si ha davanti". Per proporre allenamenti efficaci e non deleteri bisogna dunque sempre inquadrare
\begin{itemize}
    \item{Il Livello Tecnico}
    \item{Il Livello Socio-Affettivo}
    \item{Il Livello Fisico-Atletico}
\end{itemize}

In particolare risulta evidente che ad una stessa età anagrafica possono corrispondere vari gradi di maturazione in ognuno di questi tre livelli. Infatti bisogna distinguere tra vari tipi di età che non sempre coincidono:
\begin{itemize}
    \item{Età Anagrafica (o Cronologica)}\hfill \\E' data dal tempo trascorso dalla nostra nascità
    \item{Età Biologica (o Fisiologica)}\hfill \\Rappresenta l'effettiva età anatomica, funzionale ed ormonale del corpo.
    \item{Età Emotiva}\hfill \\E' il grado di capacità di provare e capire emozioni e sentimenti, propri e altrui.
    \item{Età Sociale}\hfill \\Rappresenta il ruolo che si ha nel proprio gruppo e contesto di appartenenza in senso lato.
\end{itemize}

Focalizzandoci sulla parte fisico-atletica, è dunque fondamentale capire a che punto sono i nostri bambini/e dal punto di vista dello sviluppo e della crescità andando dunque a capire l'effettiva età biologica ed il suo scarto rispetto a quella anagrafica.

\textcolor{red}{TODO}

\section{Capacità Coordinative}

\subsection{Capacità Generali}

\subsection{Capacità Specifiche}

\subsubsection{Ritmo}

La capacità di Ritmo è la capacità di cogliere un ritmo imposto dall'esterno e di riprodurlo in un gesto o in un movimento (ritmo oggettivo). Inoltre, è la capacità di realizzare i movimento secondo un andamento ritmico interiorizzato, detto ritmo soggettivo. \hfill \\
In questa capacità rientra anche il tempismo esecutivo, ovvero la capacità di eseguitre un movimento nel momento adatto e nello spazio corretto. \hfill \\
Nel calcio, esempi di questa capacità sono
\begin{itemize}
    \item Tempi e Ritmi di Gioco
    \item Variazione del Ritmo dei Movimenti a seconda della situazione
    \item Cambio di Passo
    \item Successione Regolare o Irregolare degli Appoggi
    \item Adeguamento degli Appoggi in funzione di un riferimento (palla, compagno, avversario)
\end{itemize}
In particolare, è da includere sia il movimento con l'avversario che senza, dato che il ritmo può essere derivato anche dall'ambiente circostante.

\subsubsection{Orientamento Spazio-Temporale}

E' la capacità di determinare la posizione dei propri segmenti corporei o del proprio corpo nel tempo e nello spazio, e di modificare i movimenti in relazione ai punti di riferimento (palla, compagni, avversari). \hfill \\
Nel calcio, esempi di questa capacità sono
\begin{itemize}
    \item Valutazione della Traiettoria e della Velocità della palla
    \item Valutazione della Velocità di spostamento dei giocatori (compagni e avversari)
\end{itemize}
Ad esempio, per allenare questa capacità rispetto al movimento di compagni ed avversari, si possono proporre dei giochi, esercitazioni e partite con campo a dimensioni variabili e con numero variabile di compagni e avversari. (es Small Sided Games)

\subsubsection{Differenziazione}

E' la capacità di organizzare movimenti parziali (e differenti tra loro) con un ordine cronologico-spaziale. \hfill \\
Esempi generali sono il movimento a velocità diverse e il dosaggio della forza. Nel calcio, esempi di questa capacità sono
\begin{itemize}
    \item Finte (lento veloce ad esempio)
    \item Dosaggio nella forza nei passaggi e nei tiri
    \item Ritmo di corsa e sua variazione (es contromovimenti a diverse velocità)
\end{itemize}


\subsubsection{Reazione}

E' la capacità di organizzare ed eseguire rapidamente movimenti o compiti motori coerenti in risposta ad uno stimolo, nel minor tempo possibile. \hfill \\
E' inerentemente collegata alla capacità di orientamento spazio-temporale e alla capacità di ritmo, in particolare al tempismo che possiamo declinare come:
\begin{itemize}
    \item Tempo di Azione
    \item Tempo di Movimento
    \item Tempismo Esecutivo
    \item Cambio di Passo
\end{itemize}
Nel calcio, esempi di questa capacità sono
\begin{itemize}
    \item Reazione alle Finte
    \item Tecnica del Portiere
    \item Transizioni (Positive e Negative)
\end{itemize}

\subsubsection{Combinazione e Accoppiamento}

E' la capacità di organizzare e coordinare coerentemente movimenti parziali del corpo. Questo significa ad esempio mettere assieme movimenti diversi, con e senza palla, ed eseguirli contemporaneamente o in rapida successione.
Nel calcio, esempi di questa capacità sono
\begin{itemize}
    \item Conduzione palla (Corsa) e Trasmissione
    \item Saltare e Colpire di Testa
    \item Corsa e Ricezione
    \item Controllo Orientato e Passaggio
\end{itemize}

\subsubsection{Equilibrio}

E' la capacità di mantenere il corpo stabile durante l'esecuzione di un gesto motorio. Come capacità derivata da questa prima definizione, si ha la capacità di recuperare l'equilibrio perso, ovvero di riequilibrare l'assetto corporeo durante l'esecuzione di diversi gesti motori.
Già la corsa è un esempio di questa capacità, dato che si ha un solo appoggio monopodalico per volta e una fase completamente aerea. Nel calcio, esempi di questa capacità sono
\begin{itemize}
    \item Acrobazie
    \item Colpo di Testa
    \item Cambi di Direzione
    \item Dribbling
    \item Contrasto
\end{itemize}
L'equilibrio può inoltre essere di vari tipologie:
\begin{itemize}
    \item{Statico}: Quando il corpo è fermo. Esempio è il marcamento del difensore su un attaccante fermo senza palla.
    \item{Dinamico}: Quando il corpo è in movimento. Esempio è la conduzione del pallone
    \item{Rotazionale}: Durante le rotazioni del corpo nei vari assi. Esempio è il cambio di senso.
    \item{In Volo}: Quando il corpo si trova senza appoggi. Esempi si hanno nei salti e nelle acrobazie aeree come anche nella fase di volo della corsa e nei tuffi del portiere.
\end{itemize}

\section{Capacità Biomotorie}

\subsection{Forza}

\subsection{Velocità}

\subsection{Resistenza}

\subsection{Agilità}

\subsection{Flessibilità}


\end{document}