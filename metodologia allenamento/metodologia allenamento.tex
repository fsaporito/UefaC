\documentclass[../uefaC.tex]{subfiles}

\onlyinsubfile{\author{Saporito Francesco}}
\onlyinsubfile{\title{Metodologia dell'Allenamento}}
\onlyinsubfile{\date{\today}}

\begin{document}

\onlyinsubfile{\maketitle}

\onlyinsubfile{Queste note sono una rielaborazione e approfondimento personale di quanto affrontato durante il corso Uefa C sostenuto a Sondrio da Settembre 2022 a Dicembre 2022.
Esse non rappresentano in alcun modo opinioni o dichiarazioni dei docenti del corso, della Scuola Tecnica di Coverciano, della LND, della FIGC o dell'AIAC. \textcolor{red}{TODO}
}

\onlyinsubfile{\tableofcontents}

\notinsubfile{\chapter{Metodologia dell'Allenamento}}

Durata Corso: 18h (14h Teoria in Aula, 4h Pratica sul Campo)
Esame: Tesina

\section{Maturazione}

La prima cosa che bisogna fare quando ci si affaccia all'allenamento di un giocatore o di una squadra del settore giovanile è capire "chi si ha davanti". In particolare bisogna inquadrare
\begin{itemize}
    \item{Il Livello Tecnico}
    \item{Il Livello Socio-Affettivo}
    \item{Il Livello Fisico-Atletico}
\end{itemize}

In paticolare ci focalizziamo sul livello fisico e atletico, derivato in primis dal grado di maturazione dell'atleta. Infatti bisogna distinguere tra l'età anagrafica e l'età effettiva biologica dell'atleta in considerazione.

\textcolor{red}{TODO}

\end{document}