\documentclass[../uefaC.tex]{subfiles}

\onlyinsubfile{\author{Saporito Francesco}}
\onlyinsubfile{\title{Metodologia dell'Allenamento}}
\onlyinsubfile{\date{\today}}

\begin{document}

\onlyinsubfile{\maketitle}

\onlyinsubfile{Queste note sono una rielaborazione e approfondimento personale di quanto affrontato durante il corso Uefa C sostenuto a Sondrio da Settembre 2022 a Dicembre 2022. \hfill \\
Esse non rappresentano in alcun modo opinioni o dichiarazioni dei docenti del corso, della Scuola Tecnica di Coverciano, della LND, della FIGC o dell'AIAC. \hfill \\
Queste note hanno dunque come riferimento gli appunti presi da me durante le lezioni e la bibliografia di approdondimento indicata dai vari docenti.
\textcolor{red}{TODO}
}

\onlyinsubfile{\tableofcontents}

\notinsubfile{\chapter{Metodologia dell'Allenamento}}

Durata Corso: 18h (14h Teoria in Aula, 4h Pratica sul Campo) \hfill \\
Esame: Tesina

\section{Maturazione \& Crescità}

La prima cosa che bisogna fare quando ci si affaccia all'allenamento di un giocatore o di una squadra del settore giovanile è capire "chi si ha davanti". Per proporre allenamenti efficaci e non deleteri bisogna dunque sempre inquadrare
\begin{itemize}
    \item{Il Livello Tecnico}
    \item{Il Livello Socio-Affettivo}
    \item{Il Livello Fisico-Atletico}
\end{itemize}

In particolare risulta evidente che ad una stessa età anagrafica possono corrispondere vari gradi di maturazione in ognuno di questi tre livelli. Infatti bisogna distinguere tra vari tipi di età che non sempre coincidono:
\begin{itemize}
    \item{Età Anagrafica (o Cronologica)}\hfill \\E' data dal tempo trascorso dalla nostra nascità
    \item{Età Biologica (o Fisiologica)}\hfill \\Rappresenta l'effettiva età anatomica, funzionale ed ormonale del corpo.
    \item{Età Emotiva}\hfill \\E' il grado di capacità di provare e capire emozioni e sentimenti, propri e altrui.
    \item{Età Sociale}\hfill \\Rappresenta il ruolo che si ha nel proprio gruppo e contesto di appartenenza in senso lato.
\end{itemize}

Focalizzandoci sulla parte fisico-atletica, è dunque fondamentale capire a che punto sono i nostri bambini/e dal punto di vista dello sviluppo e della crescità andando dunque a capire l'effettiva età biologica ed il suo scarto rispetto a quella anagrafica.

\textcolor{red}{TODO}

\end{document}