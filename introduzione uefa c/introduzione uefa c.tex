\documentclass[../uefaC.tex]{subfiles}

\subfilecover{Saporito Francesco}{Introduzione Uefa C}

\begin{document}

\subfileintro{Introduzione Uefa C}

\section{Scopo del Corso}

Il corso è nato per le seguenti condizioni:

\begin{itemize}
    \item{Necessità che i giovani operino con persone formate}
    \item{Volontà della FIGC di adesione alla "Carta Grassroots Uefa"}
    \item{Dare un supporto, anche tecnico, al progetto di sviluppo e valorizzazione dei vivai giovanili}
\end{itemize}

\section{Patentino Uefa C}

Il patentino Uefa Grassroots C è per molti la prima qualifica che permette di affacciarsi al campo di calcio in qualità di tecnico. \hfill \\
Non richiede di aver già sostenuto altri corsi in precedenza per l'ammissione, la quale viene valutata tramite graduatoria ottenuta considerando i punteggi in qualità di calciatore, allenatore, titoli di studio e altre certificazioni FIGC da allenatori.
\hfill \\
E' un corso erogato su base regionale o provinciale e una volta concluso con successo permette di allenare in tutte le categorie dell'attività di base e del settore giovanile agonistico escluse le squadre della primavera maschile.
\hfill \\
In particolare permette di essere tesserato come allenatore, secondo allenatore o collaboratore tecnico per le seguenti categorie dell'attività di base:
\begin{itemize}
    \item Piccoli Amici (U8)
    \item Pulcini (U10) Puri/Misti 7vs7
    \item Esordienti (U13) Puri/Misti 9vs9, Professionisti
\end{itemize}
\hfill \\
riguardo l'attività del settore agonistico maschile, permette di essere tesserato come allenatore, secondo allenatore o collaboratore tecnico per le seguenti categorie:
\begin{itemize}
\item Giovanissimi
    \begin{itemize}
        \item U14: Provinciali, Nazionali (Professionisti serie A/B/C)
        \item U15: Provinciali, Regionali, Regionali Elite, Nazionali (Professionisti serie A/B e serie C)
    \end{itemize}
\item Allievi 
    \begin{itemize}
        \item U16: Provinciali, Regionali, Nazionali (Professionisti serie A/B e serie C)
        \item U17: Provinciali, Regionali, Regionali Elite, Nazionali (Professionisti serie A/B e serie C)
        \item U18: Provinciali, Regionali, Nazionali (Professionisti serie A/B e serie C)
    \end{itemize}
\item Juniores (U19) Provinciali, Regionali B e A, Nazionali (Serie D)
\end{itemize}
\hfill \\
infine per l'attività del settore agonistico femminile permette di essere tesserato come allenatore, secondo allenatore o collaboratore tecnico per le seguenti categorie:
\begin{itemize}
    \item Giovanissime (U15) Regionali
    \item Allieve (U17) Regionali
    \item Juniores (U19) Regionali
    \item Primavera (U19) Nazionali (Serie A e B)
\end{itemize}
\hfill \\

\section{Organizzazione}
Il corso consiste in 130 ore di lezione teoriche e pratiche cosi suddivise: \hfill \\

\FloatBarrier
\begin{table}[h!]
\begin{tabular}{|c|c|c|c|}
\hline
\textbf{Materia}                      & \textbf{Ore Teoriche} & \textbf{Ore Pratiche} & \textbf{Ore Totali} \\ \hline
Tecnica e Tattica Calcistica          & 34 h                  & 24 h                  & 58 h                \\ \hline
Calcio a 5                            & 4 h                   & 2 h                   & 6 h                 \\ \hline
Calcio Femminile                      & 6 h                   & -                     & 6 h                 \\ \hline
Tecnica del Portiere                  & 6 h                   & -                     & 6 h                 \\ \hline
Teoria e Metodologia dell'Allenamento & 14 h                  & 4 h                   & 18 h                \\ \hline
Psicopedagogia                        & 6 h                   & -                     & 6 h                 \\ \hline
Medicina dello Sport                  & 6 h                   & -                     & 6 h                 \\ \hline
Regolamento di Giuoco                 & 6 h                   & -                     & 6 h                 \\ \hline
Carte Federali \& Giustizia Sportiva  & 4 h                   & -                     & 4 h                 \\ \hline
Progetto Integrato Scuola - Sport     & 2 h                   & -                     & 2 h                 \\ \hline
Tavola Rotonda                        & 2 h                   & -                     & 2 h                 \\ \hline
Incontri AIAC                         & 1 h                   & -                     & 1 h                 \\ \hline
BLSD                                  & 5 h                   & -                     & 5 h                 \\ \hline
\end{tabular}
\end{table}
\FloatBarrier
\hfill \\


Si ha un obbligo di frequenza al 90 \% e il corso si conclude con esami scritti e/o orali e/o pratici per ogni materia.



\end{document}