\documentclass[../uefaC.tex]{subfiles}

\subfilecover{Saporito Francesco}{Medicina dello Sport}

\begin{document}

\subfileintro{Medicina dello Sport}

Durata Corso: 6h (Teoria in Aula) \hfill \\
Esame: ?

\section{Cenni di Anatomia Muscolo-Schelettrica}
\textcolor{red}{TODO}

\section{Traumatologia nel Calcio}

Uno degli obbiettivi dell'attività sportiva è quello di favorire lo stato di salute, obbiettivo primario per ottenere un'ottima prestazione. \hfill \\
Lo sport deve però essere praticato "con delle garanzie" al fine di praticarlo senza correre il rischio di incorrere in incidenti. \hfill \\
\underline{Rischio}: Possibiltà di instaurarsi di un danno a seguito di circostanze, azioni, situazioni, attrezzi, definiti \emph{Pericoli} non sempre prevediibili. \hfill \\
Il rischio può essere rimosso se viene rimosso il pericolo mentre può solo essere mitigato se il pericolo permane. Ne deriva che in presenza di pericolo non può esserci rischio zero. In particolare nello sport si parla di \emph{Rischio Speciifco Sportivo} ovvero del rischio inerente nel praticare lo sport.

\textcolor{red}{TODO}

\section{Malattie dell'Età Evolutiva}

Lo sport, malgrado sia un'attività consigliata per uno stile di vita salutare e per un ottimale sviluppo dei bambini, comporta però dei rischi intrinsechi.

\textcolor{red}{TODO}

\section{Alimentazione}
Come abbiamo visto, l'attività fisica è fondamentale per il mantenimento di uno stile di vita salutare. Bisogna però associarlo ad una corretta alimentazione per ottenere i migliori risultati, sia prestativi che di prevenzione contro le malattie e in generale per il raggiungimento del benessere fisico. Questo significa non solo considerare la quantità del cibo che deve essere adeguata alla richiesta energetica dell'attività sportiva, ma anche e soprattutto alla sua qualità. \hfill \\
Infatti l'alimentazione fa parte dei 3 elementi fondamentali per dare qualità alla salute e alla vita:
\begin{itemize}
    \item Ambiente Esterno
    \item Genetica
    \item Alimentazione
\end{itemize}
La dieta del \textbf{Calciatore} deve essere in particolare \textbf{Normocalorica} e \textbf{Normoproteica} (circa 1 - 1.2 $gr / kg$), puntando sulla qualità dei cibi. Negli ultimi anni in particolare si mangiano troppo:
\begin{itemize}
    \item Pane
    \item Pasta
    \item Pizza
    \item Patate
    \item Riso
    \item Succhi di Frutta
    \item Bibite
    \item Merendine
\end{itemize}
Bisognerebbe poi togliere gli zuccheri semplici per sostituirli con quelli complessi ed eliminare i cibi spazzatura (farina 00, zucchero bianco, margarine, dolcificanti, sciroppo di fruttosio e glucosio, bibite zuccherate, alcolici, nitriti e nitrati, dolciumi industriali, glutammato, addensanti). \hfill \\
\hfill \\ 
La maggior parte delle malattie croniche (diabete, malattie cardiovascolari, tumori, Alzheimer, ...) potrebbe essere prevenuta curando l'alimentazione e la qualità del cibo.

\subsection{Colazione}
La prima colazione è il pasto più importante della giornata e non andrebbe mai saltata. In particolare deve essere ricca per dare il segnale all'ipotalamo di \emph{abbondanza}. Da qui parte un effetto a catena
\begin{enumerate}
    \item Alimentazione Ricca e Completa
    \item Produzione di \emph{Leptina} da parte del Tessuto Adiposo
    \item Stimolazione dell'ipotalamo
    \item Stimolazione Ipofisi
    \item Stimolazione Generale di Vari Organi
    \begin{itemize}
        \item Tiroide
        \item Surrene (Ghiandole Surrenali)
        \item Muscoli e Ossa
        \item Ovaio - Testicoli
    \end{itemize}
\end{enumerate}
Al mattino ci sono dunque picchi ematici di \emph{Testosterone}, \emph{Cortisolo} e \emph{Ormoni Tiroidei} che aiutano il dimagrimento mentre al pomeriggio ci sono picchi di \emph{Insulina} e \emph{GH} che invece favoriscono l'ingrassamento.

\subsection{Dieta Ideale Calciatore}
Abbiamo dunque le seguenti ripartizioni \emph{ideali} dei vari nutrimenti per la dieta del calciatore 

\begin{table}[]
    \begin{tabular}{|c|c|c|}
    \hline
    \textbf{Macronutriente} & \textbf{Giorno di Riposto} & \textbf{Giorno di Allenamento} \\ \hline
    Proteine                & 20 \%                      & 15 \%                          \\ \hline
    Grassi                  & 30 \%                      & 25 \%                          \\ \hline
    Carboidrati             & 50 \%                      & 60 \%                          \\ \hline
    \end{tabular}
\end{table}
dove la ripartizione ideale nei vari pasti dovrebbe essere:
\begin{itemize}
    \item Colazione: 30/40 \%
    \item Pranzo: 35 \%
    \item Cena: 25 \%
    \item Spuntini Vari: 5 \%
\end{itemize}

\subsection{Carboidrati e Zuccheri}
Sono praticamente nella maggior parte dei cibi che mangiamo e sono problematici perchè aumentano la \textbf{glicemia}, ovvero la concentrazione dello zucchero nel sangue il cui valore normale è circa \emph{65 - 100 mg/100 ml (2-3 gr/litro)}. In particolare per tenere controllata la glicemia si tiene conto del seguente parametro di un determinato alimento: \hfill \\
\hfill \\
\underline{Indice Glicemico}: Capacità dei Carboidrati presenti nell'alimento di innalzare la Glicemia.
\hfill \\
\hfill \\
L'aumento di Glicemia porta infatti all'aumento dell'Insulina nel corpo, da cui deriva un aumento del peso, del colesterolo e dei trigliceridi. Inoltre un alta glicemia porta al femonemo di glicazione tra una singola molecola di zucchero (es glucosio o fruttosio) con una moleica proteica o lipidica senza l'azione di controllo di un enzima, il che va ad alterare il funzionamento della proteina o del lipide, portando potenzialmente anche alla produzione di tossine. Dividiamo la glicazione in due tipologie, in base a dove si manifesta:
\begin{itemize}
    \item Glicazione Endogena: Avviene all'interno del corpo.
    \item Glicazione Esogena: Avviene all'esterno del corpo, in particolare durante la cottura o ad alte temperature sopra i 120 ° o per lunghi periodi.
\end{itemize}
Questo processo è stato correlato all'invecchiamento precoce e all'Alzheimer. In particolare nella glicazione si è visto essere molto più pericoloso il fruttosio rispetto al glicosio, avendone 10 volte la capacità glicante. \hfill \\
Particolare attenzione va anche prestata ai \textbf{Dolcificanti} che stimolano la fame e aumentano dunque l'assunzione di cibo e della glicemia.

\subsection{Verdure}
Le verdure sono un alimento fondamentale per la sana nutrizione, in quanto contengono potenzialmente tutti i nutrienti necessari alla corretta nutrizione. \hfill \\
Le verdure sono molto ricche di vitamine (nutrienti fondamentali) e di fibre, che possono dividersi in:
\begin{itemize}
    \item Fibre Solubile: Rallenta l'assorbimento di grassi e zuccheri. Hanno un'attività prebiotica ovvero promuovono una selezione di una microflora intestinale positiva.
    \item Fibre Insolubili: Si trovano nei Cereali Integrali e favoriscono il transito intestinale e riducono il rischio tumorae.
\end{itemize}
Le verdure sono poi molto ricche di \textbf{fitonutrienti} (o fotochimici) i quali, sebbene siano \emph{Nutrienti non Essenziali} per la sopravvivenza dell'uomo, hanno proprietà \underline{Antiossidanti}, \underline{Anti-Infiammatorie}, \underline{Antiallergiche}, \underline{Antivirali} e \underline{Anti-Tumorali} sono di forte aiuto nel constrasto dell'azione dei radicali liberi, dell'ossidazione e del danneggiamento delle membrane cellulari. \hfill \\
Di fondamentale importanza tra le verdure è la categoria dei \emph{Legumi}. Sono ricchi di proteine, fibre e carboidrati buoni. Sono prebiotici, antitumorali, hanno un basso indice glicemico e favoriscono l'abbassamento del colesterolo. \hfill \\

\subsection{Grassi e Olii}
I grassi vengono divisi in 3 grandi macrocategorie rispetto al loro impatto positivo o negativo sulla nostra salute:
\begin{enumerate}
    \item \textbf{Grassi Buoni}: Olio extra-vergine di Oliva, Grassi contenuti nel pesce, nelle noci, nelle mandorle, nei semi naturali (lino, sesamo), nei cereali integrali e nelle alghe.
    \item \textbf{Grassi da Limitare}: Grassi Saturi
    \item \textbf{Grassi da Evitare}: Grassi Idrogenati (Margarina), Olio di Palma, Oli Vegetali non pressati a freddo ma estratti solventi (oli di semi)
\end{enumerate}
Di particolare rilevanza è l'olio extra vergine di oliva, dato che è un antinfiammatorio naturale, contiene molti antiossidanti e ha un effetto preventivo e curativo verso molte malattie (diabete, tumori, malattie cardiovascolari). Tra i suoi benefici ci sono:
\begin{itemize}
    \item Aumenta le HDL
    \item Diminiusce le LDL
    \item Dimunuisce i trigliceridi
    \item Riduce la placca aterosclerotica
    \item Inibisce la coagulazione del sangue
    \item Ha azione antitumorale
    \item Antiaritmico
\end{itemize}
Gi oli di semi non biologici e non pressati, come quelli di lino e di girasole, vengono estratti tramite esano ad altissime temperature e hanno tutta una serie di molecole dannose e tossiche per l'uomo. Allo stesso modo l'olio di palma contiene tre sostanze tossiche (glicidiolo, 3-monocloropropandiolo, 2-monocloropropandiolo) di cui una viene classificata addirittura come genotossica e cancerogena.


\subsection{Proteine}
Le proteine sono un macronutriente fondamentale, con funzione sia strutturale che energetica. In particolare possiamo avere 3 origini diverse per le proteine che ingeriamo ovvero \emph{animale (carne)}, \emph{carne (pesce)} o \emph{vegetale}. Si è visto che un aumento di consumo di carne e di proteine animali porta ad un aumento dell'IGF1, il quale è correlato positivamente con 
\begin{itemize}
    \item Iperplasia del Tessuto Adiposo
    \item Aumento dei Tumori Intestinali (Ferro)
    \item Malattie Cardiache
    \item Osteoporosi
    \item Stitichezza
\end{itemize}
In particolare bisognerebbe cercare di limitare il consumo di proteine animali ad un massimo di 1-2 volte a settimana.

\subsection{Uova}
Le uova sono un alimento di ottime qualità per una nutrizione sana. Un uovo medio di gallina infatti contiene:
\begin{itemize}
    \item 8 gr di proteine di altissima qualità (albumina)
    \item Lipidi Saturi e Insaturi 
    \item Vitamina K2 (che favorisce la deposizione del calcio nelle ossa)
    \item Vitamine A, B, Ferro, Zinco e Calcio
\end{itemize}
Inoltre hanno un colesterolo che non ha effetti sulla colesterolemia e hanno un basso costo.

\subsection{Latte, Formaggi, Yogurt}
Contengono molti ormoni e fattori di crescita, il lattosio, la caseina e molti grassi saturi e proinfiamamtori. Portano dunque ad Osteoporosi, acne, emicrania, dermatite atopica e ad un aumento del colesterolo. Nella percezione generale sono un ottimo alimento per l'assunzione di calcio ma esso in realtà può essere assunto tramite altri alimenti quali le verdure a foglia verde (cicoria), i legumi, il pesce azzurro, la frutta secca (mandorle) e i semi di sesamo. \hfill \\
L'aumento del colesterolo è dato dall'elevato indice insulemico dei latticini e dei formaggi, il che stimola la sintesi del colesterolo a livello epatico. \hfill \\
Inoltre la \emph{Vitamina D} è fondamentale per l'assorbimento del calcio (inolte stimola le difese tumorali ed è un potente antitumorale) e si può assumere tramite i pesci, latte e formaggi. Però in realtà si assume molta più vitamina D tramite l'assorbimento dal sole che dal latte.

\subsection{Glutine}
E' una lipoproteina contenina in frumento, orzo, farro, segale e seitan. E' una specie di colla che si appiccica alle pareti dell'intestino impedendo l'effetto del sistema immunitario. I sintomi più frequenti di intolleranza al glutine sono:
\begin{itemize}
        \item Diarrea e Malassorbimento
        \item Dolori Addominali
        \item Afte ricorrenti
        \item Emicrania
        \item Osteoporosi
        \item Artralgie
        \item Dermatiti
\end{itemize}

\subsection{Vino, Birra e Alcolici}
Gli alcolici hanno una seria ripercussione sull'organismo e possono portare i seguenti effetti:
\begin{itemize}
    \item Cirrosi Epatica
    \item Pancreatite
    \item Danni Neurologici
    \item Prostatite (Birra)
    \item Tumori (derivati dall'acidificazione che porta ad uno stato infiammatorio)
\end{itemize}

\subsection{Alimentazione pre-partita al mattino}
Se si ha la partita al mattino, si consiglia una colazione composta da
\begin{itemize}
    \item The o Latte (meglio non caffelatte)
    \item Fette Biscottate o Pane Integrale con Marmellata o Miele
    \item Yogurt con Frutta o Cereali Integrali
    \item Spremuta
    \item Uovo alla Coque 
    \item Toast al Prosciutto
\end{itemize}

\subsection{Alimentazione pre-partita al pomeriggio}
Se si ha la partita al pomeriggio, si consiglia di pranzare almeno 3 ore prima, composto da:
\begin{itemize}
    \item Un piatto di pasta o riso (meglio se integrali) condito con olio extra vergine o sugo al pomodoro
    \item Verdura
    \item Porzione ridotta di Pesce o Carne (non fritti)
    \item Frutta Fresca di Stagione
    \item Acqua da bere (evitare alcolici)
    \item Evitare i dolci, soprattutto alla crema
\end{itemize}

\subsection{Alimentazione pre-partita a ridosso}
Fino ad un'ora della partita e ad intervalli regolari, introdurre
\begin{itemize}
    \item Piccole razioni di frutta fresca o disidratata
    \item Alcuni noci o mandorle
    \item Qualche biscotto secco con della Marmellata
    \item Barrete ai cereali e frutta
\end{itemize}
Sono poi indicate bevande integratrici ipotoniche con una percentuale di zuccheri non superiore al 5 \%.

\subsection{Alimentazione Durante la Partita}
Durante la partita e soprattutto all'intervallo bisogna
\begin{itemize}
    \item Non mangiare cibi solidi ma assumere solo liquidi
    \item Vanno bene soluzioni di carboidrati (maltodestrine o glucosio) con un pizzico di sale 
    \item Bere circa 200 ml ogni 15-20 minuti.
\end{itemize}

\subsection{Alimentazione Post Allenamento/Partita}
A cena bisognerebbe mangiare
\begin{itemize}
    \item Minestra di Verdura o Pasta o Riso
    \item Secondo di carne o pesce non abbondante
    \item Verdura
    \item Frutta
    \item Dolce
\end{itemize}

\end{document}