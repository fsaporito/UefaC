\documentclass[../uefaC.tex]{subfiles}

\subfilecover{Saporito Francesco}{Medicina dello Sport}

\begin{document}

\subfileintro{Medicina dello Sport}

Durata Corso: 6h (Teoria in Aula) \hfill \\
Esame: ?

\section{Cenni di Anatomia Muscolo-Schelettrica}

\section{Traumatologia nello Sport}

\section{Malattie dell'Età Evolutiva}

Lo sport, malgrado sia un'attività consigliata per uno stile di vita salutare e per un ottimale sviluppo dei bambini, comporta però dei rischi intrinsechi.

\section{Alimentazione}
Come abbiamo visto, l'attività fisica è fondamentale per il mantenimento di uno stile di vita salutare. Bisogna però associarlo ad una corretta alimentazione per ottenere i migliori risultati, sia prestativi che di prevenzione contro le malattie e in generale per il raggiungimento del benessere fisico. Questo significa non solo considerare la quantità del cibo che deve essere adeguata alla richiesta energetica dell'attività sportiva, ma anche e soprattutto alla sua qualità. \hfill \\
Infatti l'alimentazione fa parte dei 3 elementi fondamentali per dare qualità alla salute e alla vita:
\begin{itemize}
    \item Ambiente Esterno
    \item Genetica
    \item Alimentazione
\end{itemize}
La dieta del \textbf{Calciatore} deve essere in particolare \textbf{Normocalorica} e \textbf{Normoproteica} (circa 1 - 1.2 $gr / kg$), puntando sulla qualità dei cibi. Negli ultimi anni in particolare si mangiano troppo:
\begin{itemize}
    \item Pane
    \item Pasta
    \item Pizza
    \item Patate
    \item Riso
    \item Succhi di Frutta
    \item Bibite
    \item Merendine
\end{itemize}
Bisognerebbe poi togliere gli zuccheri semplici per sostituirli con quelli complessi ed eliminare i cibi spazzatura (farina 00, zucchero bianco, margarine, dolcificanti, sciroppo di fruttosio e glucosio, bibite zuccherate, alcolici, nitriti e nitrati, dolciumi industriali, glutammato, addensanti). \hfill \\
\hfill \\ 
La maggior parte delle malattie croniche (diabete, malattie cardiovascolari, tumori, Alzheimer, ...) potrebbe essere prevenuta curando l'alimentazione e la qualità del cibo.

\subsection{Colazione}
La prima colazione è il pasto più importante della giornata e non andrebbe mai saltata. In particolare deve essere ricca per dare il segnale all'ipotalamo di \emph{abbondanza}. Da qui parte un effetto a catena
\begin{enumerate}
    \item Alimentazione Ricca e Completa
    \item Produzione di \emph{Leptina} da parte del Tessuto Adiposo
    \item Stimolazione dell'ipotalamo
    \item Stimolazione Ipofisi
    \item Stimolazione Generale di Vari Organi
    \begin{itemize}
        \item Tiroide
        \item Surrene (Ghiandole Surrenali)
        \item Muscoli e Ossa
        \item Ovaio - Testicoli
    \end{itemize}
\end{enumerate}
Al mattino ci sono dunque picchi ematici di \emph{Testosterone}, \emph{Cortisolo} e \emph{Ormoni Tiroidei} che aiutano il dimagrimento mentre al pomeriggio ci sono picchi di \emph{Insulina} e \emph{GH} che invece favoriscono l'ingrassamento.

\subsection{Dieta Ideale Calciatore}
Abbiamo dunque le seguenti ripartizioni \emph{ideali} dei vari nutrimenti per la dieta del calciatore 

\begin{table}[]
    \begin{tabular}{|c|c|c|}
    \hline
    \textbf{Macronutriente} & \textbf{Giorno di Riposto} & \textbf{Giorno di Allenamento} \\ \hline
    Proteine                & 20 \%                      & 15 \%                          \\ \hline
    Grassi                  & 30 \%                      & 25 \%                          \\ \hline
    Carboidrati             & 50 \%                      & 60 \%                          \\ \hline
    \end{tabular}
\end{table}
dove la ripartizione ideale nei vari pasti dovrebbe essere:
\begin{itemize}
    \item Colazione: 30/40 \%
    \item Pranzo: 35 \%
    \item Cena: 25 \%
    \item Spuntini Vari: 5 \%
\end{itemize}

\subsection{Carboidrati e Zuccheri}
Sono praticamente nella maggior parte dei cibi che mangiamo e sono problematici perchè aumentano la \textbf{glicemia} (ovvero la concentrazione dello zucchero nel sangue). In particolare per tenere controllata la glicemia si tiene conto del seguente parametro di un determinato alimento: \hfill \\
\hfill \\
\underline{Indice Glicemico}: Capacità dei Carboidrati presenti nell'alimento di innalzare la Glicemia.
\hfill \\
L'aumento di Glicemia porta infatti all'aumento dell'Insulina nel corpo, da cui deriva un aumento del peso, del colesterolo e dei trigliceridi. Inoltre un alta glicemia porta al femonemo di glicazione tra una singola molecola di zucchero (es glucosio o fruttosio) con una moleica proteica o lipidica senza l'azione di controllo di un enzima, il che va ad alterare il funzionamento della proteina o del lipide, portando potenzialmente anche alla produzione di tossine. Dividiamo la glicazione in due tipologie, in base a dove si manifesta:
\begin{itemize}
    \item Glicazione Endogena: Avviene all'interno del corpo.
    \item Glicazione Esogena: Avviene all'esterno del corpo, in particolare durante la cottura o ad alte temperature sopra i 120 ° o per lunghi periodi.
\end{itemize}
Questo processo è stato correlato all'invecchiamento precoce e all'Alzheimer. In particolare nella glicazione si è visto essere molto più pericoloso il fruttosio rispetto al glicosio, avendone 10 volte la capacità glicante. \hfill \\
Particolare attenzione va anche prestata ai \textbf{Dolcificanti} che stimolano la fame e aumentano dunque l'assunzione di cibo e della glicemia.

\textcolor{red}{TODO}

\end{document}