\documentclass[../uefaC.tex]{subfiles}

\subfilecover{Saporito Francesco}{Medicina dello Sport}

\begin{document}

\subfileintro{Medicina dello Sport}

Durata Corso: 6h (Teoria in Aula) \hfill \\
Esame: ?

\section{Cenni di Anatomia Muscolo-Schelettrica}

\section{Traumatologia nel Calcio}

Uno degli obbiettivi dell'attività sportiva è quello di favorire lo stato di salute, obbiettivo primario per ottenere un'ottima prestazione. \hfill \\
Lo sport deve però essere praticato "con delle garanzie" al fine di praticarlo senza correre il rischio di incorrere in incidenti. \hfill \\
\underline{Rischio}: Possibiltà di instaurarsi di un danno a seguito di circostanze, azioni, situazioni, attrezzi, definiti \emph{Pericoli} non sempre prevediibili. \hfill \\
Il rischio può essere rimosso se viene rimosso il pericolo mentre può solo essere mitigato se il pericolo permane. Ne deriva che in presenza di pericolo non può esserci rischio zero. In particolare nello sport si parla di \emph{Rischio Speciifco Sportivo} ovvero del rischio inerente nel praticare lo sport.


\section{Malattie dell'Età Evolutiva}

Lo sport, malgrado sia un'attività consigliata per uno stile di vita salutare e per un ottimale sviluppo dei bambini, comporta però dei rischi intrinsechi.

\section{Alimentazione}

\textcolor{red}{TODO}

\end{document}